%+----------------------------------------------------------------------------+
%| SLIDES: 
%| Chapter: Complementary material - details on eventual questions
%| Author: Antonio miti
%| Event: PHD preliminary Defence
%+----------------------------------------------------------------------------+

%- HandOut Flag -----------------------------------------------------------------------------------------
\newif\ifHandout

%- D0cum3nt ----------------------------------------------------------------------------------------------
\documentclass[beamer,10pt]{standalone}   
%\documentclass[beamer,10pt,handout]{standalone}  \Handouttrue  

%- HandOut Flag -----------------------------------------------------------------------------------------
\ifHandout
	\setbeameroption{show notes} %print notes   
\fi

	
%- Packages ----------------------------------------------------------------------------------------------
\usepackage{custom-style}

%--Beamer Style-----------------------------------------------------------------------------------------------
\usetheme{toninus}



\providecommand{\blank}{\text{\textvisiblespace}}


\newcommand{\subsectiontitle}{
  \begin{frame}
  \vfill
  \centering
  \begin{beamercolorbox}[sep=8pt,center,shadow=true,rounded=true]{title}
    \usebeamerfont{title}\insertsectionhead\par%
    \usebeamerfont{title}\insertsubsectionhead\par%
  \end{beamercolorbox}
  \vfill
  \end{frame}
}

\providecommand{\blank}{\text{\textvisiblespace}}




%---------------------------------------------------------------------------------------------------------------------------------------------------
%- D0cum3nt ----------------------------------------------------------------------------------------------------------------------------------
\begin{document}
%------------------------------------------------------------------------------------------------

%##################################################################################
\begin{frame}
	\begin{center}
	\Huge\emph{Complementary Material}
	\end{center}
\end{frame}
\addtocounter{framenumber}{-1}
%##################################################################################





%===================================================================================
\section{Background}
%===================================================================================



%-------------------------------------------------------------------------------------------------------------------------------------------------
\subsection{Symplectic Manifolds}
%-------------------------------------------------------------------------------------------------------------------------------------------------

%-------------------------------------------------------------------------------------------------------------------------------------------------
\begin{frame}{Symplectic geometry}
\begin{columns}[T]
	\begin{column}{.50\linewidth}
		\centering
		\textit{ "geometric approach" to mechanics \dots}
		%
		\begin{columns}
			\begin{column}{.50\linewidth}
				\begin{center}
					\includegraphics[width=0.8\linewidth]{Pictures/pendulum13}			
				\end{center}
			\end{column}	
			\begin{column}{.50\linewidth}
				\begin{center}
					\includegraphics[width=0.45\linewidth]{Pictures/pendulum-phase-space}			
				\end{center}
			\end{column}	
		\end{columns}
		%
		\begin{defblock}[Symplectic Manifold]
			\includestandalone[width=0.95\textwidth]{Pictures/Figure_sym}	
		\end{defblock}
		%
		\begin{exblock}[$M = T^\ast Q$ is symplectic]
			$\omega = d \theta $ with
			$$ \left.\theta\right\vert_{(q,p)} (v) = p (\pi_\ast v) ~.$$
		\end{exblock}
	\end{column}
	\vrule{}
	\pause
	\begin{column}{.50\linewidth}
		\centering
		\textit{ "algebraic approach" to mechanics \dots}
		\vspace{1em}	
		\begin{defblock}[Classical Observables]
			Unital, associative, commutative algebra $C^\infty(M)$.
		\end{defblock}
		%
		\vspace{1em}
		\pause
		\begin{defblock}[Hamiltonian vector fields]
			$v_f \in \mathfrak{X}(M)$ such that:
			$$\iota_{v_f} \omega = -df \quad \text{(exact)}$$ %$\in B^1(M)$
			\small$v_f$ = \emph{Ham.v.f. pertaining to $f\in C^\infty(M)$}.
		\end{defblock}
		%
		\begin{defblock}[Poisson Algebra of Observables]
			$C^\infty(M)$ is a Poisson algebra with
			$$\{f,g\} = \iota_{v_g} \iota_{v_f} \omega = \omega(v_f,v_g) ~.$$
		\end{defblock}
	\end{column}
\end{columns}
\end{frame}
\note[itemize]{
	\footnotesize

	\item We work in the framework of multisymplectic geometry which is one of the possible generalizations of the well-established field of symplectic geometry.
	
	\item To recall what symplectic geometry is let me assume a particular point of view: mechanics.
	\\
	Idea:"
	Symplectic geometry is a branch of differential geometry studying symplectic manifolds; it originated as a formalization of the mathematical apparatus of classical mechanics and geometric optics."{\href{https://ncatlab.org/nlab/show/symplectic+geometry}{nlab}}
	
	Namely, a sym. mfd. is the geometric structure encoding the phase space of conservative, ordinary, classical, mechanical systems.
	
	\item $\theta$ = \emph{tautological 1-form}.
		$\theta$ evaluated at $p\in T^*Q$ in the fibre of $q\in Q$ and contracted with $v$ coincides with the form $p$ evaluated at $q$ and contracted with the push forward of $v$.
	
	\item We identify a special class of vector fields.
		Out of them one can define a Lie bracket.
	
	\item Poisson is a Lie algebra with the extra property of compatibility with the associative product (Leibniz rule)
}
%-------------------------------------------------------------------------------------------------------------------------------------------------

%------------------------------------------------------------------------------------------------
% Frame inspired by Leonid: passing from moment maps to comoment maps
\begin{frame}[fragile]{Reminder: Moment maps in symplectic geometry}
	Let $(M,\omega)$ be a symplectic manifold, $\vartheta: G\times M \to M$ a Lie group action preserving $\omega$ and $v:\mathfrak{g}\to \mathfrak{X}(M)$ the corresponding infinitesimal action
	%
		\begin{defblock}[Moment map pertaining to $\vartheta$]
			Smooth map $ \mu: M \to \mathfrak{g}^\ast$ such that: \stackunder{$d \langle\mu , \xi \rangle = \iota_{v_\xi}\omega \scriptstyle\quad \forall \xi \in \mathfrak{g}$}{$f (\vartheta_g(x)) = Ad^\ast_g (f(x)) \scriptstyle\quad \forall g \in G, x \in M$}
		\end{defblock}
	%
	\vfill
	\emph{... from a dual perspective (assuming $G$ connected) ...}
			\begin{defblock}[Comoment map pertaining to $v$]
				\begin{columns}
					\begin{column}{.5\linewidth}	
			Lie algebra morphism \qquad $ f: \mathfrak{g} \to C^\infty(M) $
			\\
			such that \qquad $ d~f (x) = -\iota_{v_x} \omega \qquad \forall x \in \mathfrak{g}~.$
					\end{column}
					\begin{column}{.4\linewidth}	
						\begin{displaymath}
							\begin{tikzcd}
								& C^{\infty}(M,\omega) \ar[d]
								\\
								\mathfrak{g} \ar[ur,dashed,"(f)"]\ar[r,"v"']& \mathfrak{X}(M)
							\end{tikzcd}	
						\end{displaymath}
					\end{column}
				\end{columns}
		\end{defblock}		
	%
	\vfill
	\emph{... a tool encoding conserved quantities ...}
	\begin{propblock}[Noether Theorem]
		\small Fixed $H\in C^\infty_{\text{Ham}}(M)$ ($\mathfrak{g}$-invariant) ,
				$$\mathcal{L}_{v_H} f(x) = 0 \qquad \forall x \in \mathfrak{g}$$
	\end{propblock}
\end{frame}
%\note[itemize]{
%	\item comoment map is a Lie algebra morphism projecting to $v$. (Triangle diagram in Lie algebra category).
%}



%------------------------------------------------------------------------------------------------
\begin{frame}{Geometry of symmetries}\label{frame:geometrysymmetries}
	Basic mechanical structures are encoded in geometry. but there is another complementary geometrical property that's natural in physics: symmetry!
	\begin{alertblock}{Upshot}
		Continous symmetries are described by actions of a Lie group on $M$.
	\end{alertblock}
	\begin{block}{Noether}
		Presence of symmetries $\quad \Rightarrow \quad$ existence of conserved quantities.
	\end{block}	
	\begin{block}{Key concept:}
		Noether current are encoded in a \emph{moment map}  $\mu :M \rightarrow \mathfrak{g}^*$ (the dual of the comoment map $f$. 
	\end{block}
  \begin{columns}[T]
   	\begin{column}{.6\textwidth}
			\begin{block}{Symplectic reduction:}
			\begin{itemize}
				\item System dynamics should be restricted to level set of conserved observables in order to efficiently store dynamical properties.
				\item Under certain assumptions, $\mu^{-1}( 0 )/G$ is a symplectic manifold with an "induced" symplectic structure.
			\end{itemize}
			\end{block}
    \end{column}
    \begin{column}{.4\textwidth}	
			\includegraphics[width=\textwidth]{Pictures/Reduction} 
  	\end{column}
	\end{columns}			
\end{frame}
%------------------------------------------------------------------------------------------------




%-------------------------------------------------------------------------------------------------------------------------------------------------
\subsection{MultiSymplectic Manifolds}
%-------------------------------------------------------------------------------------------------------------------------------------------------


%------------------------------------------------------------------------------------------------
\begin{frame}[fragile]{MS geometry and classical field mechanics}
		Consider a smooth manifold $Y$,
		\begin{columns}
			\hfill
			\begin{column}{.5\linewidth}
				\emph{Multicotangent bundle} $\bigwedge = \bigwedge^n T^\ast Y$\\
				is naturally $n$-plectic
			\end{column}
			\begin{column}{.4\linewidth}
				\[
				\begin{tikzcd}
					\Lambda \ar[d,"\pi"'] & T \Lambda \ar[d,"T \pi"] \ar[l] \\
					Y								& T Y \ar[l]
				\end{tikzcd}	
				\]
			\end{column}
		\end{columns}
	\pause
	\begin{defblock}[Tautological $n$-form]
		$\theta \in \Omega^n(\Lambda)$ such that:
		\begin{displaymath}
		\begin{split}
			\left[ \iota_{u_1 \wedge \ldots \wedge u_n} \theta \right]_\eta 
			&= \iota_{(T \pi)_\ast u_1 \wedge \ldots \wedge (T \pi)_\ast u_n} \eta \\
			&= \iota_{u_1 \wedge \ldots \wedge u_n} \pi^\ast \eta 
			\qquad \qquad \forall \eta \in \Lambda \, , \: \forall u_i \in T_\eta \Lambda 		
		\end{split}
		\end{displaymath}
	\end{defblock}
	\vfill
	\begin{columns}
		\begin{column}{.6\linewidth}
			\begin{defblock}[Tautological (multisymplectic) (n+1)-form]
				$$\omega := d \theta$$
			\end{defblock}
		\end{column}
		\begin{column}{.4\linewidth}
		 	\begin{claimblock}$\omega$ is not degenerate.\end{claimblock}	
		\end{column}
	\end{columns}	
	\pause
	\begin{keywordblock}
		\begin{tabular}{|c|c|c|}
			\hline 
			point-particles mechanics & $\rightsquigarrow$ & classical fields mechanics \\
			%(finite discrete DOF) & & (finite dimensional continuous DOF) \\
			\hline 
			symplectic & $\rightsquigarrow$ & multisymplectic \\ 
			\hline 
			Observables (Poisson) algebra & $\rightsquigarrow$ & Observables $L-\infty$ algebra
			 \\ 
			\hline 
			Co-moment map & $\rightsquigarrow$ & Homotopy co-momentum map \\ 
			\hline 
		\end{tabular} 
	\end{keywordblock}

	
\end{frame}
\note[itemize]{
	\item This example is significant from the perspective of geometric classical field theory:
		\begin{displaymath}
			\frac{\text{classical mechanics}}{\text{symplectic geo.}} =
			\frac{\text{classical field mechanics}}{\text{multisymplectic geo.}}
		\end{displaymath}
	\item Multicotangent bundle is the \emph{Higher analogue} of the cotangent bundle.
	(but it is not yet the analogue of a \emph{phase space}.)
\item The multiphase space is the sub-bundle of $n$-forms vanishing when contracted with 2 vertical fields.
  	\item The reason why this sub-bundle has a particular role is that it can be proved to be isomorphic to a suitable dual of the first Jet bundle.
  	\item For further details see Gotay et al. \href{https://arxiv.org/abs/physics/9801019}{arXiv:physics/9801019}. For a pictorial representation of all the structures involved in the geometric mechanics of I order classical field theories see appendix, pag: \ref{frame:Gimmsy}.
}
%------------------------------------------------------------------------------------------------

%------------------------------------------------------------------------------------------------
  \begin{frame}[fragile]{GIMMSY construction} \label{frame:Gimmsy}
  		\includestandalone[width=0.90\textwidth]{Pictures/Figure_ms_landscape}  	
  \end{frame}
  \note{}
%------------------------------------------------------------------------------------------------

	
	
	
%------------------------------------------------------------------------------------------------
\begin{frame}{Special classes of smooth objects} 
  	\begin{columns}
		\begin{column}[t]{.42\linewidth}		
			\begin{defblock}[Hamiltonian v.f.]
				$\mathfrak{X}_{ham} =  \left\lbrace X \in  \mathfrak{X} \right\vert \left. \iota_x \omega \textrm{ exact}  \right\rbrace$ 			
			\end{defblock}
			\begin{defblock}[Multisymplectic v.f.]
				$\mathfrak{X}_{ms} =  \left\lbrace X \in  \mathfrak{X} \right\vert \left. \mathcal{L}_X \omega = 0  \right\rbrace$ 	
			\end{defblock}
		\end{column}
		\begin{column}[t]{.58\linewidth}		
			\begin{defblock}[Hamiltonian $(n$-$1)-$forms]
				\begin{displaymath}
					\Omega^{n-1}_{ham} 	:=
					\biggr\{ H \in  \Omega^{n-1} \; \left\vert \; 
					\stackanchor{$\exists X \in \mathfrak{X}_{ham}$}{: $d H = -\iota_X \omega$} \right\} 
			\end{displaymath}
			\end{defblock}		
		\end{column}
  	\end{columns}
  	%
  	\vspace{0.5em}
  	%
  	\onslide<2->{
  	\begin{columns}
		\begin{column}[t]{.5\linewidth}	
			\centering\emph{Global symmetries}
			\begin{defblock}[Multisymplectic (Lie group) action]
				$\Phi: G \circlearrowright (M, \omega)$ \emph{right action} s.t. \\
				$$\hat{\Phi}(g)_\ast \omega = \omega \quad \forall g \in G$$
			\end{defblock}
		\end{column}
		\begin{column}[t]{.5\linewidth}			
			\centering\emph{Infinitesimal symmetries}
			\begin{defblock}[Multisymplectic (Lie algebra) action]
				$V: \mathfrak{g} \rightarrow \mathfrak{X} (M)$ \emph{Lie algebra morphism} s.t. \\
				$$\mathcal{L}_{V_\xi} \omega = 0 \quad \forall \xi \in \mathfrak{g}$$	
			\end{defblock}
		\end{column}
  	\end{columns}
  	}
  	%
  	\onslide<3->{		
	  	\begin{asideblock}[Hierarchy of conserved quantities]%Shades of...
	  		\begin{table}[] % http://tablesgenerator.com/
			\begin{tabular}{lllll}
					& strictly conserved & & & $\mathcal{L}_X \alpha= 0$ \\
				$\alpha \in \Omega^\bullet$ & globally conserved & along $X \in \mathfrak{X}$ & $\Leftrightarrow$ & $\mathcal{L}_X \alpha\in B $ (exact) \\
				  & locally conserved  & & & $\mathcal{L}_X \alpha\in Z $ (closed)                                
			\end{tabular}
			\end{table}
	  	\end{asideblock}
  	}
  	
  \end{frame}
  \note[itemize]{
  	\item Exactly as it happens in symplectic geometry, fixing a smooth form $\omega$ on $M$ yields a criterion for classifying vector fields and differential forms.
  	\\(Pay attention to the sign convention in defining the Hamiltonian vector fields)
  	\item Also, we can naturally select a special class of symmetries (global and infinitesimal) which preserve the fixed multisymplectic form.
  	\item Aside, we can start to see that, in this setting, measurable quantities are not only smooth functions but also differential forms with degree greater then zero.
  	For such objects can be defined weaker notions of conservation along a flow.
  	\item The idea to consider forms of various degree as observables do not fall out of the sky. 
  		For instance in a string there will be two kind of measurable quantities: extensive observable (1-forms), like the density, and intensive observables (0-forms), like the tension. 
 		%\href{https://en.wikipedia.org/wiki/Intensive_and_extensive_properties#Intensive_properties}{(wiki link on this terminology)}
  	\item Starting from this observation we can define the space of all possible observables (see next slide).
  }
%---------------------------------------------------------------------------------------------------------------------------------------------------


%-------------------------------------------------------------------------------------------------------------------------------------------------
\subsection{$L_\infty$-algebras}
%-------------------------------------------------------------------------------------------------------------------------------------------------

%---------------------------------------------------------------------------------------------------------------------------------------------------
\begin{frame}[fragile,shrink]{Unwrapping the \emph{higher Jacobi equations}}\label{Frame:unwapping-Jacobi}
\underline{Slogan:} \emph{Jacobi identity satisfied up to an higher coherent homotopy}
		%
		\vspace{1.5em}
		\begin{columns}[c]
			\hfill
			\begin{column}{0.5\linewidth}
				Higher Jacobi implies:
				\begin{itemize}  \setlength\itemsep{1em}
					\item Underlying chain-complex $(L,\mu_1)$ with differential $d=\mu_1$;
					\item \color{red} $\mu_2 = [\cdot,\cdot]$ is a chain map $L^{\otimes 2} \to L$;
					\item \color{green!20!black}$\mu_3=j(\cdot,\cdot,\cdot)$ is a chain homotopy 
						$\mu_2\circ\mu_2 \Rightarrow 0$;
						\\ i.e. between the usual Jacobiator ${[[\cdot,\cdot],\cdot]} \circ P_{\text{unsh}}$ and the $0$ map 
					\item \color{purple}higher analogues...	
					\\ e.g. $\mu_4$, is a second order chain-homotopy between the two chain homotopies  ${[j(\cdot,\cdot ,\cdot]),\cdot]}\circ P_{\text{unsh}}$ and ${j([\cdot , \cdot],\cdot,\cdot)}\circ P_{\text{unsh}}$
				\end{itemize}
			\end{column}
			\begin{column}{0.45\linewidth}
				\includestandalone[width=0.9\linewidth]{Pictures/Figure_Linfinity_diagram}
			\end{column}	
		\end{columns}	
		\vspace{1.5em}
		Notation: $P_{\text{unsh}}$ = sum on all the possibile unshuffled permutation.

\end{frame}
\note[itemize]{
  \item Regarding any $l_k$ as a tree with $k$ entries and 1 output, the $k$-th generalized Jacobi equation is produced summing all the possible way to obtain a $k+1$-ary tree by composing two other trees (not more then two!).
  \item Can be regarded as
  	\begin{displaymath}
  		\sum_{i+j = k} l_j \circ ( l_j \otimes \mathbb{I}) \circ P_{\text{unsh}}
  	\end{displaymath}
  	Where $P_{\text{unsh}} : L^{\otimes(k-1)} \rightarrow L^{\otimes(k-1)} $ is the $(i,j)$-unshuffolator.
  	\\(you consider only unshuffles to avoid the redundancies given by the fact that any $l_i$ has fixed symmetry.
  \item Examples of unshuffles: \\
  \begin{displaymath}
  \begin{split}
  	(12)(3)\quad(13)(2)\quad(23)(1)\\
  	(123)(4)\quad(234)(1)\quad(134)(2)\quad(124)(3)\\
  	(12)(34)\quad(23)(14)\quad(13)(24)\quad(14)(23)\quad(24)(13)
  \end{split}
  \end{displaymath}
	\item When regarding the L$\infty$ structure as a chain complex with homotopies you get a neat intepretation of the Jacobi identity at the price that \emph{graded skew-symmetry} definition is more obscure than in the presentation with graded vector spaces.
}
%------------------------------------------------------------------------------------------------



%-------------------------------------------------------------------------------------------------------------------------------------------------
\subsection{Observables $L_\infty$-algebra}
%-------------------------------------------------------------------------------------------------------------------------------------------------

%------------------------------------------------------------------------------------------------
\begin{frame}{Lie $\infty$-algebra of Observables \emph{(Rogers)}}
	\begin{defblock}[$L_\infty$-algebra \emph{(Lada, Stasheff) \cite{Lada1993}}]
		\includestandalone{Pictures/Figure_Linfinitydef}
	\end{defblock}	
	%
	\pause
	\vfill
	\begin{thmblock}[Rogers \cite{Rogers2010}]
		The \emph{higher observable algebra} $L_{\infty}(M,\omega)$ 	forms an honest $L_\infty$ algebra.
		\footnotetext{Take $[\cdot]_1$ equal to the deRham differential.}
	\end{thmblock}
\end{frame}
\note[itemize]{
	\item $L_\infty$-algebra is the notion obtained from a Lie algebra requiring that the Jacobi identity is satisfied only up to a higher coherent chain homotopy.
	\item The Lie-n algebra mentioned before is a $L_\infty$ algebra with underlying graded vector space concentrated in degrees $0,1...n$.
	
	\item Definition. We say that a permutation $\sigma \in S_n$ is a $(j,n-j)$-unshuffle, $0\leq j \le1 n$  if $\sigma(1)< \dots < \sigma(j)$ and $\sigma(j+1)<\dots<\sigma(n)$.
	\\
	You can also say that $\sigma$ is a $(j,n-j)$-unshuffle if $\sigma(i)< \sigma(i+1)$ when $i\neq j$.

	\item 	Alternatively, the Jacobiators can be also denoted as $$\displaystyle J_m=\sum_{i+j=m+1}(-)^{i(j+1)} 	\mu_i \circ \mu_j = 0$$
	employing the so-called \emph{ Richardson-Nijenhuis product}
		 $\mu_i\circ \mu_j := \frac{1}{j!(i-1)!}\mu_i\cdot\mu_j \cdot \mathcal{A}~, \qquad \mathcal{A} =$ (graded) total skew-symmetrizator.
		 
	\item see frame extra-\ref{Frame:unwapping-Jacobi} for a slightly demystification of the higher Jacobi equations.

	\item more precisely this statement is a proposition/definition

}
%------------------------------------------------------------------------------------------------


%------------------------------------------------------------------------------------------------
%Slide by Leonid
\begin{frame}[fragile]{Why the $L_\infty$ algebra of observable is so "simple"}
	\textit{(in the sense that the higher brackets are defined only on $L^0$, i.e. "grounded")}
	 \\
	 \vfill
	Extend the underlying cochain complex
	\begin{displaymath}
		\begin{tikzcd}
			C^{\infty}(M) \ar[r,"d"] &
			\cdots \ar[r,"d"] &
			\Omega^{n-2}(M) \ar[r,"d"] &
			\Omega^{n-1}_{Ham}(M,\omega) \ar[r,dashed] &
			\mathfrak{X}_{Ham}(M,\omega)
		\end{tikzcd}
	\end{displaymath}
	Consider 
	\begin{displaymath}
		\begin{tikzcd}[column sep= small,row sep=0ex]
			\{\cdot,\cdot\}_2 ~\colon&[-1em] \left(\Omega^{n-1}_{\textrm{Ham}}(M,\omega)\right)^{\otimes 2} 	\arrow[r]& 				\Omega^{n-1}(M) \\[-1ex]
			& \sigma_1\otimes\sigma_1 	\ar[r, mapsto]& 	-
			\iota_{\mathscr{v}_{\sigma_1}}\iota_{\mathscr{v}_{\sigma_2}}\omega 
		\end{tikzcd}
	\end{displaymath}
	%
	\vfill
	\begin{thmblock}[Barnich, Fulp, Lada, Stasheff \cite{Barnich1998}]
		 Let $L^\bullet = (\cdots \rightarrow L^{-1} \rightarrow L^0 \rightarrow \mathfrak{g})$ be a resolution of the Lie
		algebra $\mathfrak{g}$.
		\begin{itemize}
			\item a skew symmetric $\ell_2:L^0\times L^0 \to L^0$ covering the Lie bracket of $\mathfrak{g}$ can extended to a $L_\infty$-algebra structure $\{\ell_k\}$ on $L_\bullet$.
			\item If $\ell_2$ is zero on boundaries, then the structure can be chosen such that $\ell_i$, for $i\geq 2$, are non-zero only on $L^0$.
		\end{itemize}
	\end{thmblock}


\end{frame}
\note[itemize]{
	\item resolution in the sense that the $0$-th homology group is isomorphic to $\mathfrak{g}$ and all other cohomology groups are trivial (\cite[\S 2.1]{Barnich1998})
	\item the extended complex is a resolution only if $M$ is contractible. In other terms, such a resolution exists locally on any multisymplectic smooth manifold.

}



%------------------------------------------------------------------------------------------------


%-------------------------------------------------------------------------------------------------------------------------------------------------
\subsection{Homotopy comoment maps}
%-------------------------------------------------------------------------------------------------------------------------------------------------


\begin{frame}{Comoment maps}
	Consider a Lie algebra action $v:\mathfrak{g} \to \mathfrak{X}(M)$  preserving the $n$-plectic form $\omega$,
	\vfill
	\begin{columns}
		\begin{column}{.5\linewidth}	
	\textbf{Symplectic case $(n=1)$}
		\begin{defblock}[Comoment map pertaining to $v$]
			Lie algebra morphism
			$$ f: \mathfrak{g} \to C^\infty(M) $$
			such that
			$$ d~f (x) = -\iota_{v_x} \omega \qquad \forall x \in \mathfrak{g}~.$$
		\end{defblock}		
		\end{column}
		\begin{column}{.5\linewidth}	
	\textbf{Multi-symplectic case $(n\geq 1)$}
		\begin{defblock}[Homotopy comoment map \tiny (HCMM)]
			$L_\infty$-morphism 
			$$ (f_k) : \mathfrak{g} \to L_\infty (M,\omega)$$
			such that
			$$ d~f_1(x) = -\iota_{v_x} \omega \qquad \forall x \in \mathfrak{g}~.$$
		\end{defblock}		
		\end{column}
	\end{columns}	
	%
	\pause
	\centering \textbf{-- Conserved quantities --}
	%
	\begin{columns}
		\begin{column}{.5\linewidth}		
			\begin{propblock}[Noether Theorem]
				\small Fixed $H\in C^\infty_{\text{Ham}}(M)$ ($\mathfrak{g}$-invariant) ,
				$$\mathcal{L}_{v_H} f(x) = 0 \qquad \forall x \in \mathfrak{g}$$
			\end{propblock}
		\end{column}
		\begin{column}{.5\linewidth}			
			\begin{propblock}[RWZ16 Theorem]
				\small Fixed $H\in \Omega^{n-1}_{\text{Ham}}(M)$ ($\mathfrak{g}$-invariant),
				$$\mathcal{L}_{v_H} f_k(p) \in B^k(M) \qquad \forall p \in Z_k(\mathfrak{g})$$			
			\end{propblock}
		\end{column}
	\end{columns}



\end{frame}
\note[itemize]{
	\item  An infinitesimal symmetry is a lie algebra morphism such that $\mathcal{L}_{v_x} \omega = 0 ~ \forall x \in \mathfrak{g}$.
	\\ (It is also call an infinitesimal multisymplectic action and $v_x$ is the infinitesimal generator of the action, corresponding to $x \in \mathfrak g$.) 
	\item Essentially, admitting a comoment maps mean that $v$ acts via Hamiltonian vector fields.
	\item In mechanical terms, a moment map is a tool associated with a Hamiltonian action of a Lie group on a symplectic manifold, used to construct conserved quantities for the action.
	\item The name derives from the special case given by angular momentum in the dynamics of rigid bodies, 
	\item Notation [RWZ16]: H is called \emph{strictly invariant} and $f_k(p)$ are \emph{globally invariant}.
	\\
	$B^k(M)$ are exact differential k-forms and $Z_k(\mathfrak{g}$ are Eilenberg Chevalley homology k-cycles.
	
	\item Details about Reduction in frame \ref{frame:geometrysymmetries} of the  appendix.
	
}
%-------------------------------------------------------------------------------------------------------------------------------------------------

%------------------------------------------------------------------------------------------------
  \begin{frame}[fragile,t]{Chevalley-Eilenberg Complex \qquad\hyperlink{frame:hcmm-main}{\beamerreturnbutton{}}}\label{frame:CE-complex}
  	Consider $\mathfrak{g}$, Lie Algebra.
  	\begin{defblock}[Eilenberg-Chevalley Complex]
  		Chain Complex
			\begin{center}
				\begin{tikzcd}[column sep= small,row sep=0.25ex]
					\ldots \ar[r,"\partial"] & \wedge^k \mathfrak{g} \ar[r,"\partial"] & 
					\wedge^{k-1} \mathfrak{g} \ar[r,"\partial"] & \ldots
			\end{tikzcd}	
			\end{center}
			with chain group
			\begin{displaymath}
				C^k := \wedge^k \mathfrak{g} \equiv 
				\big\{ c : \mathfrak{g}^\ast\times\ldots\mathfrak{g}^\ast \to \mathbb{R}\:\big\vert\, \textrm{alternating, k-linear} \big\}
			\end{displaymath}
			and boundary operator defined as
			$\partial \equiv \partial^k :  \Lambda^{k} {\mathfrak g} \to \Lambda^{k-1} {\mathfrak g}$  via
			$$
				\partial (\xi_1 \wedge \xi_2 \wedge \dots \wedge \xi_k) := \sum_{1\leq i< j \leq k} (-1)^{i+j}\, [\xi_i, \xi_j] \wedge \xi_1 \wedge \dots {\hat \xi}_i \wedge \dots \wedge {\hat \xi}_j \wedge \dots \xi_k
			$$
			where $\hat{}$ denoting deletion and with $\partial_0 = 0$.
  	\end{defblock}
		\begin{claimblock}
			$$\partial^2 = 0$$
		\end{claimblock}		
  \end{frame}
 % \note{{frame:hcmm-main}}
%----------------------------------------------------------------------------------------------


%-------------------------------------------------------------------------------------------------------------------------------------------------
\begin{frame}[fragile]{Homotopy co-moment maps \emph{(Callies, Fregier, Rogers, Zambon)}}
	Consider a multisymplectic action $G \circlearrowright (M, \omega)$,
	\pause
	\begin{lemblock}[HCMM unfolded  \cite{Callies2016}]
			%
			HCMM is a sequence of (graded-skew) multilinear maps:
			\begin{displaymath}
				(f)  = \big\lbrace f_k: \; \Lambda^k{\mathfrak g} \to L^{1-k} \subseteq \Omega^{n-k} 
				\;\big\vert\; 0\leq k \leq n+1  \big\rbrace
			\end{displaymath}
			%
			\vspace{-.5em}	
			\includestandalone[width=0.9\textwidth]{Pictures/Frame_HCMM}
			
			\vspace{-1em}		
			\emph{fulfilling:}%\emph{such that:}
			\begin{itemize}
				\item<2-> $f_0 = 0 $, $f_{n+1} = 0$
				\item<3-> $d f_k (p) = f_{k-1} (\partial p)  - (-1)^{\frac{k(k+1)}{2}} \iota(v_p) \omega 
				\qquad\scriptstyle \forall p \in \Lambda^k(\mathfrak{g}),\; \forall k=1,\dots n+1$
			\end{itemize}
		\end{lemblock}



\end{frame}
\note[itemize]{
	%\item 		Consider:  $v:\mathfrak g\to \mathfrak X(M)$  a Lie algebra morphism  s.t. $\mathcal{L}_{v_x}\omega=0 \quad  \forall x\in\mathfrak g$ (i.e infinitesimal multisymplectic Lie algebra action $\mathfrak{g}\circlearrowleft (M,\omega)$)
	\item More conceptually, a comoment is an $L_\infty$-morphism $(f):\mathfrak{g}\to L_\infty(M,\omega)$ lifting the action $v:\mathfrak{g}\to \mathfrak{X}(M)$, 
i.e. making the diagram commute in the $L_\infty$-algebras category.
	\item The vertical arrow is the trivial $L_\infty$-extension of the function mapping any Hamiltonian form to the unique corresponding Hamiltonian vector field (an it is zero elsewhere)
		\\
		(Note that any Lie algebra can be seen as an $L_\infty$-algebra concentrated in degree $0$, therefore any $L_\infty$-morphism $L\to\mathfrak{g}$ is simply given by a linear map $L_0 \to \mathfrak{g}$ preserving the binary brackets.)
	\item We will make use of an explicit version of this definition which is expressed by the lemma.
	 Practically speaking, a HCMM is given by several multilinear maps ...
	 \item In the equation we have tacitly set $\Lambda^{-1}(M) = 0$
	 %\item Notation: \qquad $\partial =$ Chevalley-Eilenberg boundary operator.
	%\item Notice that a HCMM pertains to an "infinitesimal" action of ${\mathfrak g}$ on $M$ with ${\mathfrak g}$ being the Lie algebra of a generic Lie group $G$, acting on $M$ by $\omega$-preserving vector fields.
		\item (Notation) $ p = \xi_1 \wedge \xi_2 \wedge \dots \wedge \xi_k$, 
			then $v_p = v_1 \wedge v_2 \wedge \dots \wedge v_k$ 
			where $v_i \equiv v_{\xi_i}$ are the fundamental vector fields associated to the action $G \circlearrowright M$.
	%	\item (Notation) $\iota(v_p) \omega = \iota(v_k)\dots\iota(v_1) \omega$
	%	\item $\varsigma(k) := - (-1)^{\frac{k(k+1)}{2}}$ 
		\item (Notation) $(\iota^{k}_{\mathfrak{g}}\omega)(p):= \iota(v_p) \omega = \iota(v_k)\dots\iota(v_1) \omega$
		\item $\partial \equiv \partial_k:  \Lambda^{k} {\mathfrak g} \to \Lambda^{k-1} {\mathfrak g}$  is the usual Eilenberg-Chevalley complex boundary operator (see appendix, pag: \ref{frame:CE-complex});
%		\item The definition tells us that the {\it closed} forms
%			$$\mu_k := f_{k-1} (\partial p) +  \varsigma(k) \iota(v_p) \omega 	$$
%			must actually be {\it exact}, with potential $-f_k(p)$.  	
		\item The last equation tells us that an HCMM is not a chain complex morphism but is rather a chain complex homotopy between 0 and the multicontraction $\alpha=(\iota^{k}_{\mathfrak{g}}\omega)$ (see next slide).
		is a chain map by lemma 2.18 \cite{Ryvkin2016}).
}
%---------------------------------------------------------------------------------------------------------------------------------















%===================================================================================
\section{Foreground}
%===================================================================================

%-------------------------------------------------------------------------------------------------------------------------------------------------
\subsection{Work Done with Leonid}
%-------------------------------------------------------------------------------------------------------------------------------------------------
\subsectiontitle


%-------------------------------------------------------------------------------------------------------------------------------------------------
\begin{frame}[fragile]{Cohomological obstruction to HCMM \qquad\hyperlink{frame:introcohoobstruction}{\beamerreturnbutton{}}}\label{frame:AppCohomoObstructions}
	Let $(M,\omega)$ multisymplectic and $\vartheta: M \times G \to M$ preserves $\omega$. 
	\vfill
	A HCMM is a sequence of linear maps:
	\vspace{-.5em}
	\begin{displaymath}
		(f)  = \big\lbrace f_k: \; \Lambda^k{\mathfrak g} \to L_{k-1} \subseteq \Omega^{n-k} 
		\;\big\vert\; 0\leq k \leq n+1  \big\rbrace
	\end{displaymath}	
	\vfill		
	\emph{It can be interpreted as primitives of a certain cocycle in the total complex of}
	\vspace{-.5em}
	\begin{displaymath}
		\Big(C_\mathfrak{g}^{\bullet,\bullet} = \Lambda^{\geq 1} 
		\mathfrak{g}^*\otimes \Omega^\bullet(M) 
		\cong Hom(\Lambda^\bullet \mathfrak{g},\Omega^\bullet),~\delta_\text{CE},~d\Big)
		~,	
	\end{displaymath}
	\vfill
	%
	\pause
	\begin{propblock}[$\vartheta ~\text{admits HCMM}~ ~\Longleftrightarrow~ \lbrack\tilde{\omega}\rbrack=0\in H^{n+1}(C_\mathfrak g^\bullet,d_ {tot})$	
	]
		Where:
		\begin{columns}
		\begin{column}{.5\textwidth}
			\begin{displaymath}
				\tilde{\omega} = \sum_{k=1}^{n+1} (-1)^{k-1} \iota^k_\mathfrak{g} \omega \in C_\mathfrak{g}^{n+1},
			\end{displaymath}		
		\end{column}
		\begin{column}{.5\textwidth}
			\begin{align*}
				\iota^k_\mathfrak{g} \colon \Omega^\bullet(M)
				&\to \Lambda^k \mathfrak{g}^\ast \otimes \Omega^{\bullet-k}(M)
				\\ \omega&\mapsto \omega_k = 
				\left(p \mapsto \iota_{v_p} \omega  \right) ,
			\end{align*}
		\end{column}		
		\end{columns} 
	\end{propblock}
	%
\end{frame}
\note[itemize]{
	\item 
	The  corresponding total complex is given by
	\begin{displaymath}
		(C_\mathfrak{g}^{\bullet}, d_\text{tot} = 
		\delta_\text{CE}\otimes \text{id} + \text{id}\otimes d),
	\end{displaymath}
	where $d$ denotes the de Rham differential and $\delta_{CE}:\Lambda^k\mathfrak g^*\to \Lambda^{k+1}\mathfrak g^*$ the Lie algebra cohomology differential.
	According to the Koszul sign convention, $d_{\text{tot}}$ acts on an element of $\Lambda^k \mathfrak{g}^*\otimes \Omega^\bullet(M)$ as $\delta_\text{CE} + (-1)^k d$.
	
	\item the multicontraction $\alpha=(\iota^{k}_{\mathfrak{g}}\omega)$ 	is a chain map by lemma 2.18 \cite{Ryvkin2016}).

	\item PROP: the primitives of the natural cocycle $\tilde{\omega}$ are in one-to-one correspondence with comoments of $v$.
	

}
%-------------------------------------------------------------------------------------------------------------------------------------------------




%-------------------------------------------------------------------------------------------------------------------------------------------------
\begin{frame}[fragile]{Cohomological obstructions for compact groups \qquad\hyperlink{frame:obstructioncompactgroups}{\beamerreturnbutton{}}}\label{frame:cohomologicalproposition}
	Let $\vartheta:G\times M\to M$ be a compact Lie group acting on a pre-multisymplectic manifold, preserving the pre-multisymplectic form $\omega$. 
	%
	\begin{propblock}
		[ $\exists$ (HCMM) $ 
			~\Leftrightarrow~ 
			\lbrack\vartheta^*\omega-\pi^*\omega\rbrack=0\in H^{n+1}(G\times M)$]
		Based on the sequence of isomorphisms:
		\begin{center}
			\begin{tikzcd}
	 			\Omega^\bullet(M,\vartheta) \ar[d,"\vartheta^\ast-\pi^\ast"] &\quad
				 & H_\text{dR}(M) \ar[d,"\vartheta^\ast-\pi^\ast"]  
				 & \lbrack \omega \rbrack \ar[d,mapsto]
				 \\ 
				 \Omega^\bullet(G\times M, r\times id) \ar["\cong",leftrightarrow]{d} &\quad
				 & H_\text{dR}(G\times M) \ar[leftrightarrow,"\cong"]{d}[swap]{\text{\tiny (K\"unneth)}} 
				 & \lbrack \vartheta^\ast \omega - \pi^\ast \omega \rbrack \ar[ddd,mapsto]
				 \\ 
				 \Omega^\bullet(G,r) \otimes \Omega^\bullet(M) \ar["\cong",leftrightarrow]{d}[swap]{} &\quad
				 & H_\text{dR}(G) \otimes  H_\text{dR}(M) \ar[d,"\cong",leftrightarrow]
				 \\ 
				 \Lambda^\bullet \mathfrak{g}^* \otimes \Omega^\bullet(M)\ar["\cong",leftrightarrow]{d} &\quad
				 & H_\text{CE}(\mathfrak{g}) \otimes  H_\text{dR}(M) 
				 \ar["\cong",leftrightarrow]{d} & 
				 \\
				 C_\mathfrak{g}^\bullet \oplus ( \mathbb{R}\otimes \Omega^\bullet(M))&\quad 
				 & H(C_\mathfrak{g}^\bullet)\oplus H_\text{dR}(M)
				 & \lbrack \tilde{\omega}\rbrack
			\end{tikzcd}
		\end{center}						
	\end{propblock}
\end{frame}
%-------------------------------------------------------------------------------------------------------------------------------------------------



%-------------------------------------------------------------------------------------------------------------------------------------------------
\begin{frame}[fragile]{Hints to the relation with equivariant cohomology \qquad\hyperlink{frame:obstructioncompactgroups}{\beamerreturnbutton{}}}\label{frame:EquivariantCohomology}
	\begin{itemize}
	\item 	Let a compact Lie group $G$ act on a manifold $M$. Let $EG$ be a contractible space on which $G$ acts freely by $\vartheta^{EG}$. Then we define the equivariant cohomology of $M$ as $H^\bullet_G(M):=H^\bullet((M\times EG)/G)$, where $G$ acts on $M\times EG$ diagonally.

	\item As $EG$ might not be a manifold, we have to interpret $H^\bullet(\cdot)$ as a suituable cohomology theory (e.g. singular cohomology with real coefficients) in the above definition.
	
	\item As $G$ is compact, when $\vartheta:G\times M\to M$ is a free action, we have $H_G^\bullet(M)=H^\bullet_{dR}(M/G)$. For a not necessarily free action $\vartheta$, we still have the following diagram
	$$
		G\times (M\times EG) \xrightarrow{\vartheta\times \vartheta^{EG}}
		M\times EG \xrightarrow{q} (M\times EG)/G
	$$
	where $q$ is the projection to the orbits, which induces $q^*$ in cohomology.	
	\end{itemize}
		\vfill
	%
	Let $G\times M\to M$ be a compact Lie group preserving a pre-multisymplectic form $\omega$.
	%
	\begin{tcolorbox}
	\begin{itemize}
		\item[\CheckedBox]  Proved, without resorting on a specific model, that if $[\omega]\in H^\bullet(M)$ lies in the image of $q^*:H^\bullet_G(M)\to H^\bullet(M)$, then $\vartheta$ admits a comoment.
	\end{itemize}
	\end{tcolorbox}










	


\end{frame}
\note[itemize]{


	\item We gave	an intrinsic proof of Theorem which does not depend on the choice of a model for equivariant cohomology. 
	


}
%-------------------------------------------------------------------------------------------------------------------------------------------------


%-------------------------------------------------------------------------------------------------------------------------------------------------
\subsubsection{HCMM for actions on spheres}
%-------------------------------------------------------------------------------------------------------------------------------------------------

%-------------------------------------------------------------------------------------------------------------------------------------------------
\begin{frame}{HCMM for actions on spheres \underline{(sketch of proof)}\qquad\hyperlink{frame:leoresults}{\beamerreturnbutton{}}}\label{frame:LeoThmProof}
	%Idea of proof:
	\begin{itemize}
	\setlength\itemsep{1em}
	%
		\item Obstructions live in: \quad 
			\begin{minipage}[t]{.5\textwidth}
			 \vspace{-2em}
			 \ifHandout
				{
				   \begin{alignat*}{3}
					 &H^n(C_\mathfrak{g}) 
					 :=
					 &&
					 {\phantom{\oplus}~ \cancel{H^{n-1}(S^n)\otimes H^1(G)}~\oplus}
					 \\
					 & &&
					 {\oplus~ \cancel{H^{n-2}(S^n)\otimes H^2(G)}~\oplus}
					 \\
					 & &&
					 \qquad \vdots
					 \\
					 & &&
					 \oplus {\mathbb{R}}\otimes H^n(G) \phantom{\oplus}
					 &&\onslide<2->{\cong H^n(G)}
				\end{alignat*}		
				}			 
			 \else
				{
				   \begin{alignat*}{3}
					 &C_\mathfrak{g}^n 
					 :=
					 &&
					 \only<1>{\phantom{\oplus} H^{n-1}(S^n)\otimes H^1(G)\oplus}
					 \only<2->{\phantom{\oplus} \cancel{H^{n-1}(S^n)\otimes H^1(G)}\oplus}
					 \\
					 & &&
					 \only<1>{\oplus H^{n-2}(S^n)\otimes H^2(G) \oplus}
					 \only<2->{\oplus \cancel{H^{n-2}(S^n)\otimes H^2(G)} \oplus}
					 \\
					 & &&
					 \qquad \vdots
					 \\
					 & &&
					 \oplus \only<1>{H^{0}(S^n)}\only<2->{\mathbb{R}}\otimes H^n(G) \phantom{\oplus}
					 &&\onslide<2->{\cong H^n(G)}
				\end{alignat*}		
				}						 
			 \fi
			\end{minipage}
		\item<3-> \textcolor{blue}{Lemma:} $\vartheta ~\text{admits HCMM}~ ~\Leftrightarrow~  \vartheta_p^\ast[\omega]= 0 \in H^n(G)$	
			\begin{flushright}
				( for some orbit map $\vartheta_p:G\to S^n,~ g \mapsto g \cdot p$)
			\end{flushright}
	\end{itemize}
	%
	\vfill
	\begin{itemize}
	\setlength\itemsep{1em}
	%		
		\item<4-> \textcolor{blue}{\textbf{Intransitive case:}} there exists an orbit $O = p\cdot G \subset S^n $ of $dim < n $,
			\begin{flushright}
				$\Rightarrow \vartheta^\ast_p[\omega] = \vartheta^\ast_p[\omega\vert_O] = 0$ (by dimensional reasons)
			\end{flushright}
		
		\item<5-> \textcolor{blue}{\textbf{Transitive case:}} all possible compact effective actions are classified
\begin{table}[]
\begin{tabular}{ll}
 $SO(n)/SO(n-1) = S^{n-1}$ & $G_2/ SU(3) = S^6$ \\
 $SU(n)/SU(n-1) = S^{2n-1}$ & $Spin(7)/G_2 = S^7$ \\
 $Sp(n)/Sp(n-1) = S^{4n-1}$ & $Spin(9)/Spin(7) = S^{15}$
\end{tabular}
\end{table}
	prove whether or not $\vartheta_N^\ast[\omega]$ is a generator (thus nonzero) in $H^n(G)$.
	\\
	\small
	(use \emph{Leray-Hirsch} theorem on $H \hookrightarrow G \twoheadrightarrow S^n$ for the action $G/H=S^n$.)
	\end{itemize}
	
\end{frame}
\note[itemize]{
	\item Recall that the cohomology of the sphere is non zero only in the top and the zero degree.
	\item Restricting to the action of compact groups  allows us to take advantage of the following isomorphism
	$$ H_{dR}(G) \cong H(G,r) \cong H_{CE}(\mathfrak{g})$$ coming from the "avarage trick" for compact groups.
}
%-------------------------------------------------------------------------------------------------------------------------------------------------

%-------------------------------------------------------------------------------------------------------------------------------------------------
\subsubsection{Examples}
%-------------------------------------------------------------------------------------------------------------------------------------------------
%-------------------------------------------------------------------------------------------------------------------------------------------------
\begin{frame}[fragile]{HCMM for spheres \underline{(Intransitive example)} \qquad\hyperlink{frame:leoresults}{\beamerreturnbutton{}}}\label{frame:LeoNonTransExample}
	\begin{claimblock}
		Explicit HCMM for $SO(n) \circlearrowleft \left( S^{n}, \omega\right)$ \qquad\qquad $\scriptstyle (\forall n \geq 2)$	
	\end{claimblock}
	%
	\vfill
	\begin{center}
			\begin{tikzcd}[every matrix/.append style={draw, inner ysep=0pt},column sep= small,row sep=0ex]
				f_i \colon&[-4ex] \Lambda^i\mathfrak{so}(n) \arrow[r]& \Omega^{n-1-i}(S^n)\\
				& q 	\ar[r, mapsto]& -j^\ast\iota(v_q)(\iota_E \beta)	
			\end{tikzcd}	
	\end{center}
	%
	\vfill
	\begin{columns}[T]
		\begin{column}{.6\linewidth}
		\quad where: %$\quad \bullet \quad E$ Euler vector field
	\begin{itemize}[<+->]
		\setlength\itemsep{1em}
				\item $E$ Euler vector field
				\item $j:S^n\hookrightarrow \mathbb{R}^{n+1}$ (standard) embedding
				\item $\iota_{v_{\xi_1\wedge\dots\wedge \xi_n}} = \iota_{v_{\xi_n}}\dots \iota_{v_{\xi_1}} \quad \scriptstyle ( \forall \xi_i \in \mathfrak{so}(n))$
				\item $\beta = (\hat{\varphi}~x^0)~ d x^1\wedge\dots\wedge d x^n$ 
				\\
				{\tiny ($x^0,\dots x^n$ standard Euclidean coordinates in $\mathbb{R}^{n+1}$.)}
				\item 
						$
						\displaystyle
						\hat{\varphi}(x,r)  = 
						\begin{cases}
								\frac{\left(x (n+1) - r \arctan\left(\frac{x}{r}\right)\right)}
								{\left((x)^2 + r^2\right)^{\frac{n+1}{2}}}
								&~\scriptstyle r \neq 0
								\\
								\frac{(n+1)}{x^n} 
								&~\scriptstyle r=0,~ x\neq 0
						\end{cases}
				$
	\end{itemize}

			\end{column}	
	  	\hfill  	
			\begin{column}{.35\linewidth}
					\onslide<5->{
						\begin{figure}[c]
							\vspace{1.5em}
							%https://www.wolframalpha.com/input/?i=%28x*%285%2B1%29-r*%28atan%28x%2Fr%29%29%29*%281%2F%28x**2%2Br**2%29%29**%28%285%2B1%29%2F2%29
							\href{http://shorturl.at/hxFR5}{\includegraphics[width=\textwidth]{Pictures/Figure_primitivePlot.png}}
	  					\caption{\small $\hat{\varphi}\in C^\infty (\mathbb{R}^{n+1}\setminus{0})$ \\ in cylindrical coordinates $(x=x^0,r)$}
						\end{figure}		
					}
				%
	 	 	\end{column}
 	 \end{columns}






	


\end{frame}
\note[itemize]{
	\item
}
%-------------------------------------------------------------------------------------------------------------------------------------------------

%-------------------------------------------------------------------------------------------------------------------------------------------------
\begin{frame}[fragile]{HCMM for spheres \underline{(Transitive example)} \qquad\hyperlink{frame:leoresults}{\beamerreturnbutton{}}} \label{frame:TransExample}

	\begin{claimblock}[]
	Explicit HCMM for $SO(n+1) \circlearrowleft \left( S^{n}, \omega\right)$ \qquad($n$ even)
	\end{claimblock}
	\begin{columns}
	%
		\begin{column}{.4\linewidth}
			\begin{itemize}
				\item			Choose basis in $\mathfrak{so}(n)$:	
			\end{itemize}		
		\end{column}
		%
		\begin{column}{.65\linewidth}
	\[
	A_{a b} =(-)^{1+a+b}\left(
	\resizebox{.35\linewidth}{!} 
		{
			\bordermatrix{
								& &\scriptstyle a & & \scriptstyle b & \cr
               	& 0 &    & \cdots &  & 0 \cr
              \scriptstyle a	& 0  &  \cdots & 0 & 1 & 0\ \cr
               	& 0 &   & \cdots &  & 0\cr
              \scriptstyle b	& 0 & -1 & 0 & \cdots & 0 \cr
               	& 0  &   & \cdots &  & 0 }
		}
  \right)
  \]		
		\end{column}
	\end{columns}
	%
	\pause
	\vfill
	\begin{columns}
		\begin{column}{.4\linewidth}
			\begin{itemize}
				\item 			Fundamental vector fields:
			\end{itemize}
		\end{column}
		\begin{column}{.65\linewidth}
			\[v_{A_{a b}}=  (-1)^{1+a+b}\left(x^a \partial_b - x^b \partial_a\right)\]
		\end{column}	
	\end{columns}
	\pause
	\vfill
	\begin{columns}
		\begin{column}{1.05\linewidth}
			\begin{itemize}
				\item From the vanishing of $H^1(G),H^2(G)$ one gets:
				\begin{empheq}[box=\fbox]{align*}
					f_1 (A_{a b}) &= 
					- \iota(v_{F^1(A_{a b})}) \omega =
				\dfrac{1}{n-2}\sum_{k=1}^n \iota(v_{A_{k b}})\iota(v_{A_{k a}})\omega.				
					\\
					f_2(A_{a b} \wedge A_{c d}) &= 
					\dfrac{-1}{4}
						\sum_{k=1}^n	
					\left(
						\iota(v_{A_{k a} \wedge A_{k b} \wedge A_{c d}}) - 
						\iota(v_{A_{a b}\wedge A_{k c} \wedge A_{k d}})
					\right)\omega\\
					f_2(A_{j a} \wedge A_{j b}) &= \dfrac{-1}{n-2}
					\sum_{k=1}^n	
					\left(
						\iota(v_{A_{k a}\wedge A_{j b}\wedge A_{k j}})
					\right)\omega.
				\end{empheq}
				\pause
				\item 	Construction of $f_k$ with $k\geq 3$ can be addressed numerically.
			\end{itemize}
		\end{column}
	\end{columns}



\end{frame}
\note[itemize]{
	\item 	
	First 2 component of HCMM for $SO(n+1)$ on $S^{n}$ (Going higher can be set up as a computational problem - we have a sketch of code in Python -, the first 2 component are easier due to the vanishing of the first two CE cohomology group of $G$)

	\item Recall that $\mathfrak{so}(n)$ is the Lie sub-algebra of $\mathfrak{gl}(n,\mathbb{R})$ consisting of all skew-symmetric square matrices. A basis can be constructed as follows:
\begin{equation}\label{eq:standard-basis}
	\mathcal{B}\coloneqq \big\lbrace 	A_{a b} = (-1)^{1+a+b} \left( E_{a b} - E_{b a}\right)
	\quad \vert \quad 1\leq a<b\leq n \big\rbrace
\end{equation}
where $E_{a b}$ is the matrix with all entries equal to zero and entry $(a,b)$ equal to one.
\item The fundamental vector field of $A_{a b}$ associated to the linear action of $SO(n)$ on $\mathbb{R}^n$ reads as follows:
\begin{displaymath}
	v_{A_{a b}}= \sum_{i,j}[A_{a b}]_{i j}x^j \partial_i  = (-1)^{1+a+b}\left(x^a \partial_b - x^b \partial_a\right)
\end{displaymath}
	\item from the vanishing of the first cohomology group of $(SO(n))$ one can give the action of $f_1$ and of $f_2$ (given on the only two subset of generators such that the image does not vanish)
	\item \textbf{$f_k$ for any $SO(n)$ and $k\geq 3$}:\\
	We know from Theorem \ref{thm:son-cohomology} that $H^3(\mathfrak{so}(n))$ never vanishes... how can we proceed? Tony drafted a code in Python. The repository is still private
}
%-------------------------------------------------------------------------------------------------------------------------------------------------









%===================================================================================
\subsection{Work Done with Marco}
%===================================================================================









%===================================================================================
\subsection{Work Done with Mauro}
%===================================================================================

%------------------------------------------------------------------------------------------------
  \begin{frame}{Hydrodynamical homotopy co-momentum map (details 1) \qquad\hyperlink{frame:hydro2}{\beamerreturnbutton{}}}\label{frame:hydromomap-details}
  	\begin{claimblock}
  		Explicit construction of an HCMM for $SDiff_0 \circlearrowright (\mathbb{R}^3,\nu)$
  	\end{claimblock}
	\begin{columns}
		\begin{column}[c]{.5\linewidth}
		  	\begin{itemize}
		  		\item The observables are  $$L= \Omega^1_{\textrm{ham}}(\mathbb{R}^3)\oplus\Omega^0(\mathbb{R}^3)$$
		  		\item HCMM consists of a pair of functions:
					\begin{align*}
						f_1 &\colon \mathfrak{g} \rightarrow \Omega^1_{\textrm{ham}}(\mathbb{R}^3) \\
						f_2 &\colon \mathfrak{g}\wedge\mathfrak{g} \rightarrow C^\infty(\mathbb{R}^3)
					\end{align*}	
		  	\end{itemize}
		\end{column}	
	  	\hfill  	
		\begin{column}[c]{.5\linewidth}
  		\includestandalone[width=\textwidth]{Pictures/Figure_Euclid_Trigger}
 	 	\end{column}
 	 \end{columns}
 	\begin{columns}
		\begin{column}[c]{.8\linewidth}
		 	 \begin{itemize}
				\item Satisfying the following system:
					\begin{displaymath}
						\begin{cases}
							\textrm{d} f_1(\xi) = \iota_\xi \nu = -\alpha^1(\xi) \\
							\textrm{d} f_2(\xi_1 \wedge \xi_2) = f_1\left([\xi_1,\xi_2]\right) - \iota_{\xi_2}\iota_{\xi_1} \nu 
							 := \mu_2(\xi_1,\xi_2)\\
							f_2\left(\partial \xi_1 \wedge \xi_2 \wedge \xi_3 \right) = \iota_{\xi_3}\iota_{\xi_2}\iota_{\xi_1} \nu
						\end{cases}
					\end{displaymath}
		 	 \end{itemize}
 	 	\end{column}
		\begin{column}[c]{.2\linewidth}
 	 	\end{column}
 	 \end{columns}
  \end{frame}
	\note[itemize]{
		\item (Regarding the diagram)
		\item On the left there is the part of the Chevalley-Eilemberg complex that interact with the L-$\infty$ algebra of observables.
		\item On the right there is the whole de Rham complex of the manifold $M=\mathbb{R}^3$.
		\item Even if only $\Omega^1$ and $\Omega^0$ take part in the definition of a $HCMM$, the Riemmanian structure determine a correspondence with the rest of the de Rham complex.
		\item In order to give an HCMM for this action is necessary to give a solution of the system of 3 equations below.
		\item Recall: $ 	\ast: \Omega^k \rightarrow \Omega^{n-k}$ where $\ast \sigma$ is defined as the unique form such
		 that $ \omega \wedge \ast \sigma = \nu \lbrace \omega, \sigma \rbrace$ where 
		 $\langle,\rangle$ is the inner product on forms induced by the metric. 
	}  
%---------------------------------------------------------------------------------------------------------------------------------------------------


%--------------------------------------------------------------------------------------------------------------------------------------------------- 
  \begin{frame}[t]{Explicit Construction of the hydrodynamical HCMM \qquad\hyperlink{frame:hydro2}{\beamerreturnbutton{}}}
		\begin{enumerate}
			\item<1-> Fix $\vec{b} \in \mathfrak{g}$ and define $f_1(b) := -\vec{A}^\flat$.\\ 
				The first equation is equivalent to solve
				\begin{displaymath}
					\tag{equation of magnetostatic}
					{\rm curl}(\vec{A})=\vec{b}
				\end{displaymath}
				This equation admits a solution 
				\begin{displaymath}
					\tag{Biot-Savart law}
					\vec{A}(r) = \int\frac{\vec{b}\times(\vec{r}-\vec{r}')}{|\vec{r}-\vec{r}'|^3}\textrm{d}r'
				\end{displaymath}							
				\alert {Defined up to a gradient \emph{(gauge freedom)}}. 
			%
			\item<2-> $\mu_2(\xi_1,\xi_2)$ is closed $\forall \xi\in\mathfrak{g}$ $\xRightarrow[\text{lemma}]{\text{Poincar\'e}}$ is exact.\\
				%Hence it is also exact \emph{(Poincar\'e lemma)}.\\
				Take as $f_2(\xi_1,\xi_2)$ a primitive $0$-form, \alert{determined up to a constant $c(\xi_1,\xi_2)$}.
			%
			\item<3-> Third equation is a priori only true up to a constant $c(\xi_1, \xi_2, \xi_3)$.\\
				The constant is zero since $\nu(\xi_1, \xi_2, \xi_3)$ vanishes at infinity and 
				the same is true for $f_2(\partial q)$ upon solving the related Poisson equation
 				\begin{displaymath}
 			 		\Delta f_2(\partial q) = \Delta \nu(\xi_1, \xi_2, \xi_3)			
 				\end{displaymath}
		\end{enumerate}
  \end{frame}
  \note[enumerate]{
  	\item	
  		\begin{itemize}
		  	\item Exploiting the correspondence between tangent fields and 1-form given by the metric is possible to recast the first equation in a simple vector calculus equation containing  the curl.
		  	\item Such equation is the well-known equation of magnetostatic with admits solution by the Biot-Savart law.
		  	\item In the context of hydrodynamic $A$ can be interpreted as a \emph{velocity field} and $b$ as the corresponding vorticity. (See slide for further details)  		
  		\end{itemize}
  	\item In the language of physics, such primitive form is often called "a potential".
  	\item 
  		\begin{itemize}
  			\item (Notation) $q = \xi_1 \wedge \xi_2 \wedge x_3$.
  			\item Last equation tells us that $f_2(\partial q)$ and $ \nu(q)$ differs by a constant. But since both of them vanish at infinity this constant has to be zero.
  			\item The condition of vanishing at infinity has been imposed in order to fulfil the last equation.
  		\end{itemize}
  	\item[$\triangleright$] This construction can be generalized to oriented Riemannian manifolds with further cohomological condition. 
  		See appendix, pag: \ref{frame:RiemannianGeneralization}.
  }
%---------------------------------------------------------------------------------------------------------------------------------------------------


%---------------------------------------------------------------------------------------------------------------------------------------------------
\begin{frame}[fragile]{Hydrodynamics reinterpretation \qquad\hyperlink{frame:hydro2}{\beamerreturnbutton{}}}\label{frame:HydroHCMM-reinterpretation}
	\label{frame:hydro-reinterpretation}
		Consider the loop spaces $L{\mathbb R}^3$,\\
		%
		\begin{propblock}[HCMM for $G\circlearrowright(\mathbb{R}^3,\nu)$ induces \\an ordinary co-mo.map for $G\circlearrowright (LS,\nu^{\ell})$]
			The HCMM $f \colon \mathfrak{g} \to L_{\infty}(\mathbb{R}^3,\nu)$ previously given
			\emph{transgresses}	 to
			\begin{displaymath}%\tag{Arnol'd-Marsden-Weinstein\\ hydrodynamical co-momentum map}
				\begin{tikzcd}[column sep= small,row sep=0ex]
					\lambda \colon& \mathfrak{g}	\arrow[r]& C^\infty(LS) \\
					& {\mathbf b}	\arrow[r, mapsto]& \displaystyle \lambda_b(\textvisiblespace) =-\oint_{\textvisiblespace} A^\flat = - \oint_{\textvisiblespace} f_1({\mathbf b}) 
					%	\quad \forall \gamma \in LS		
				\end{tikzcd}	
			\end{displaymath}
			that is a  moment map for the induced action $G$ on the pre-symplectic loop space $(LM,\nu^{\ell})$. (Smooth space in the sense of Brylinski)
		\end{propblock}
		\begin{itemize}
			\item $\lambda$ corresponds to \emph{Arnol'd-Marsden-Weinstein hydrodynamical co-momentum map}  defined on $\infty$-dim. manifolds.
			\item<2-> $\Lambda = \left\lbrace \lambda_{\mathbf b} \right\rbrace_{{\mathbf b}\in\mathfrak{g}}$ is, up to sign, the {\it Rasetti-Regge current algebra}
			\item<3-> There is a naturally defined {\it Poisson brackets} on $\Lambda$:
				\begin{displaymath}
					\{ f_1({\mathbf b}), f_1({\mathbf c}) \} (\cdot):= \iota_{\mathbf c} \iota_{\mathbf b} \nu (\cdot)=
						\nu({\mathbf b}, {\mathbf c}, \cdot) = f_1([{\mathbf b},{\mathbf c}])
					-df_2 ({\mathbf b} \wedge {\mathbf c})
				\end{displaymath}
				\centering\footnotesize(Note: $\lambda$ is (infinitesimally) $G$-equivariant, i.e. $	\{\lambda_{\mathbf b}, \lambda_{\mathbf c} \} = \lambda_{[{\mathbf b}, {\mathbf c}]}$)
		\end{itemize}

    
\end{frame}
\note[itemize]{
  	\item[ ] \textbf{How all of this is relevant in Hydrodynamics?}
  	\item The loop space is the manifold, in the sense of Brylinsky, consisting of all smooth loops in ${\mathbb R}^3$.
  	\item Transgression can be seen as a pull-back along the evaluation map 
  		$${\rm ev}: L{\mathbb R}^3 \times {\mathbb R} \ni (\gamma, t) \mapsto \gamma(t) \in {\mathbb R}^3$$
  		  	For further details see appendix, pag: \ref{frame:LoopSpacesTransgression}.
  	\item Note that the (RR) current pertaining to ${\mathbf b} \in {\mathfrak g}$  is independent of the choice of $B$.
  			See appendix, pag: \ref{frame:RRcurrents} for other informations on this concept or \cite{Rasetti1975},\cite{Penna1992} for a deeper account.
  	\item In \cite{Callies2016} is proved a general result asserting that, roughly speaking,
			homotopy co-momentum maps transgress to homotopy co-momentum maps on loop (and even mapping) spaces. Further details in appendix, pag: \ref{frame:TransgressionHCMM}.
	\item Actually, the ansatz for $f_1$ term in the previous construction has been precisely motivated by this phenomenon. 
}
%------------------------------------------------------------------------------------------------




%------------------------------------------------------------------------------------------------
\begin{frame}{Non-Equivariance of the Hydrodynamical HCMM \qquad\hyperlink{frame:hydro2}{\beamerreturnbutton{}}} 
	%
 	\begin{columns}
		\begin{column}[c]{.5\linewidth}	
			\begin{defblock}[$G$-equivariant HCMM]
				$f_i\colon \Lambda^i\mathfrak g\to \Omega^{n-i}(M)$ \\ 
				Ad(G)-equivariant $ \forall i\in\{1,...,n\}$.
			\end{defblock}		
		\end{column}
		\begin{column}[c]{.5\linewidth}	
			\begin{defblock}[$\mathfrak{g}$-equivariant HCMM]
				$\mathcal{L}_{v_x}(f_i(q))=f_i([x,q])$ \\
				\phantom{\hspace{2cm}}$\forall q\in \Lambda^{i}\mathfrak g , \: x\in \mathfrak g$\\
				\footnotesize{Where $[x,\cdot]$ is $ad(x)$ acting  on $\Lambda^\bullet \mathfrak g$}
				%$f_i\colon \Lambda^i\mathfrak g\to \Omega^{n-i}(M)$
			\end{defblock}		
		\end{column}
	\end{columns}
	\pause
	\vfill
	\begin{propblock}[$(f)$ is not $\mathfrak{g}$-equivariant]
		\underline{Proof:} 
		\footnotesize
		Consider in particular $\xi = \vec{b} \in \mathfrak{g}$, one should check 
		\begin{displaymath}
			\begin{split}
				0 = f_1\left( [\xi, \vec{b}] \right)
				\stackrel{?}{=}
				%=?=
				{\mathcal L}_{\xi} f_1({\vec{b}})  = -{\mathcal L}_{\xi} \vec{A}^\flat &=
					- d \iota_\xi \vec{A}^\flat = - d \langle \vec{A}, \vec{b} \rangle_g
				\qquad \forall \xi \in \mathfrak{g} \\
						\text{(vanishing at infinity condition)} &\Rightarrow \langle \vec{A}, \vec{b} \rangle_g = 0
			\end{split}
%		\begin{split}
%					f_1\left( [\xi, \vec{b}] \right) &= 0 \\
%					 ( \forall \xi\in \mathfrak{g}) \quad \parallel ? & \qquad\\
%					{\mathcal L}_{\xi} f_1({\vec{b}})  &= -{\mathcal L}_{\xi} \vec{A}^\flat =
%					- d \iota_\xi \vec{A}^\flat = - d \langle \vec{A}, \vec{b} \rangle_g
%		\end{split}		
		\end{displaymath}
		%
	 	\begin{columns}
			\begin{column}[c]{.7\linewidth}
				(By contradiction), consider $\vec{b}\in\mathfrak{g}$ supported on two linked flux tubes
				$(\text{supp}(\vec{b}) = \Gamma_1 \cup \Gamma_2)$, one gets
		\begin{displaymath}
			\int \langle \vec{A}, \vec{b} \rangle_g = 2 \mathbf{n} \Phi_1 \Phi_2 
			\qquad \text{with} \quad 
			\Phi_i = \int_{S_i} \hat{n} \cdot \vec{b}\, d\sigma
		\end{displaymath}
		where $n \in \mathbb{N}$ \emph{(Gauss linking number)}, results
		$$\mathbf{n} \neq 0 \qquad \Rightarrow \quad {\mathcal L}_{\xi} f_1({\vec{b}}) \neq 0  \quad \Rightarrow \quad\text{\faBomb}$$ 
			\end{column}
			\begin{column}[c]{.45\linewidth}	
				\includegraphics[width=0.9\linewidth]{Pictures/VortexLink}
			\end{column}
		\end{columns}	
	\end{propblock}

		
\end{frame}
\note[itemize]{
    \item We may also naturally ask the question whether the above map $(f)$ is (infinitesimally) {\it G-equivariant}, in the sense of \cite{RWZ}.
    \item In the symplectic case, $G$ connected implies $G$-equivariance. This does not hold true in the multisymplectic case. (ask Leyli!)
    \item In the proof, if $\vec{A}$ is the velocity field of the fluid, $\vec{b}$ is the associated vorticity, the quantity $\langle \vec{A}, \vec{b} \rangle_g$ is called \emph{Helicity}, see below (pag. \ref{Frame:VortexLinks}) for further details.
    \item The proof is by contradiction by exhibiting an example of solenoidal field $\vec{b} = \text{curl}(\vec{A})$ with non vanishing helicity, i.e. not satisfying the condition.\\
    Namely, consider a solenoidal field compactly supported  on a domain $D$ consisting in two disjoint, unknotted but linked closed tubes.
    \item  See \cite{Moffatt-Ricca92} for further elucidation on the non vanishing of the Helicity in the cosidered case.
    Notice that the argument does not depend on the choice of	$\vec{A}$ pertaining to the fixed $\vec{b}$.
    \item The lack of $G$-equivariance is not surprising, since our construction involves Riemannian geometric features.	
}
%------------------------------------------------------------------------------------------------



%------------------------------------------------------------------------------------------------
  \begin{frame}[shrink]{Riemannian Generalization \qquad\hyperlink{frame:hydro2}{\beamerreturnbutton{}}}\label{frame:RiemannianGeneralization}
			\begin{columns}
				\begin{column}{.5\linewidth}
					\begin{itemize}
						\item  $(M,g)$ be a connected compact oriented Riemannian manifold of dimension $n+1$
						\item such that the {\it de Rham} cohomology groups $H_{dR}^{k}(M)$ vanish for $k=1,2,\dots n-1$ 
						\item endow it with the multisymplectic form $\nu$ given by its Riemannian volume form
						\item consider $g_0$, lie algebra consisting of divergence-free vector fields vanishing at a point $x_0 \in M$.
					\end{itemize}
					\begin{claimblock}
						$\exists (f)$, family of HCMM, pertaining to the infinitesimal action of $\mathfrak{g}_0$ on $M$
					\end{claimblock}
				\end{column}
				\begin{column}{.5\linewidth}
 			 		\includestandalone[width=\textwidth]{Pictures/Figure_Riemann_Trigger}
				\end{column}
			\end{columns}
		\textit{Sketch of proof:}\\
			The defining formula triggers a recursive construction starting from $f_1$.\\
			Take $ f_1(\xi) := -\Delta^{-1} \delta (\iota_{\xi} \nu)$
			$\quad$ (...) (see \cite{Miti2018})
  \end{frame}
  		\note[itemize]{
			%\item[(Thm)] A hydrodynamically flavoured HCMM can be similarly construed also for an $(n+1)$-dimensional connected, compact, orientable Riemannian manifold $(M,g)$, upon taking its Riemannian volume form $\nu$ as a multisymplectic form and again the group $G$ of volume preserving diffeomorphism group as symmetry group.
			\item[(recall)] The divergence of a vector field $X$ is defined via ${\rm div}\, X := *d\!*\!X^{\flat} = -\delta X^{\flat} $.
			\item The recursive construction start from $f_1$, up to topological obstructions  since we have a sequence of closed forms, which must be actually exact, together with the constraint $ f_n(\partial q) = (-1)^{\frac{(n+1)(n+2)}{2}}\nu(\xi_1,\dots\xi_{n+1})$, with $q= \xi_1 \wedge \dots \xi_{n+1}$, for the {\it constant} function $\mu_{n+1}(\cdot)$.
				\item A natural candidate for the (n-1)-form $f_1$ can be manufactured via Hodge theory as 
				$f_1(\xi) := -\Delta^{-1} \delta (\iota_{\xi} \nu)$
			after imposing $\delta f_1({\xi}) = 0$ (the analogue of the Coulomb gauge condition).
			\item Condition $H_{dR}^{n-1}(M) = 0$ assure one can safely invert the Hodge Laplacian $\Delta = d\delta + \delta d$;
			\item The 	topological assumptions made ensure that the entire procedure goes through unimpeded due to the formula
			$$
			df_k (\xi_1 \wedge\dots \wedge \xi_k) = \mu_k (\xi_1 \wedge\dots \wedge \xi_k), \qquad \qquad k=2,3,\dots n
			$$
			\item Finally, one has to check that 
			$$
			f_n (\partial (\xi_1 \wedge\dots \wedge \xi_{n+1})) =  -\varsigma(n+1) \iota (\xi_1 \wedge\dots \wedge \xi_{n+1})\nu
			$$
			this is guaranteed by the vanishing of all fields at $x_0$.
			% this is true once we notice that, since $c_{x_0} = 0$, the class $[c_x] = 0$ ( see \cite{Callies2016}, section 9).\qed
		}
%------------------------------------------------------------------------------------------------  



%------------------------------------------------------------------------------------------------
\begin{frame}{A connection with Knot theory \qquad\hyperlink{frame:hydro3}{\beamerreturnbutton{}}}\label{Frame:VortexLinks}
 	\centering$\Rightarrow$\emph{The bridge is the \alert{Vortex Dynamics}}$\Leftarrow$

	\begin{columns}
	\begin{column}[T]{.55\linewidth}
		\vspace{1ex}
 		$\bullet$Consider a perfect (incompressible, inviscid) fluid permeating the  whole space $M=\mathbb{R}^3$.\\
 		\pause
 		$\bullet$ $\text{Physical state} \rightsquigarrow u \in sdiff_0(\mathbb{R}^3)$
% 		\begin{displaymath}
% 			\text{physical state} \rightsquigarrow u \in sdiff_0(\mathbb{R}^3)
% 		\end{displaymath}
 		\begin{columns}
 		\begin{column}{.5\linewidth}	
 			\begin{defblock}[Vorticity]
				$\displaystyle \omega := \textrm{curl}(u)$					
 			\end{defblock}
 		\end{column}
  		\begin{column}{.5\linewidth}	
 			\begin{defblock}[Helicity]
				$\displaystyle h = \int_{\mathbb{R}^3} u \cdot \omega  d^3x$		
 			\end{defblock} 		
 		\end{column}
 		\end{columns}
 				%
  		\begin{columns}
 		\begin{column}{.5\linewidth}	
 			\pause
			$\bullet$ Dynamics is ruled by the \emph{Euler equation}:
			\begin{displaymath}
				\frac{\partial \omega}{\partial t} = [\omega, u]				
			\end{displaymath}
 		\end{column}
  		\begin{column}{.5\linewidth}	
			\begin{propblock}[Helicity is conserved]
				$\frac{d}{dt}H = 0$
			\end{propblock}
 		\end{column}
 		\end{columns}
 			\pause
 			\vspace{2ex}
 			$\bullet$
 			Consider $\omega$ localized in a flux loop $\mathcal{L}$.
 			\begin{propblock}[Vortex filaments are preserved]
 				$\text{supp}(\omega(0))\simeq \text{supp}(\omega(t)) \qquad \forall t$
  			\end{propblock}
	\end{column}
	\begin{column}[T]{.40\linewidth}
		\vspace{-1ex}
		\pause
		\begin{figure}[T]
			\caption{Knotted vortex in water  (Klenecker \& Irvine \cite{Kleckner2013})}
			\href{https://www.nature.com/articles/nphys2560}{\includegraphics[width=\linewidth]{Pictures/knottedwatervortex}}
		\end{figure}
			\vspace{-2ex}
			\alert{Such fluid configurations are proved to exist both 
			\underline{theoretically} and \underline{experimentally}!}
	\end{column}
	\end{columns} 
\end{frame}
\note[itemize]{
	\item One of the  the main goal of our paper\cite{Miti2018} is to find an application of this construction in knot theory.
	\item The cornerstone of this idea is to recognize the ubiquitous role of the group of preserving diffeomorphisms. 
	The first step is to notice that conserved quantities can be associated to any fluid configurations (especially knotted ones).
 	\item The gist of the second proposition is that the topology of the support of the vorticity is conserved along the fluid evolution.
 	\item The image on the right represents a fluid configuration with knotted vorticity realized in the laboratory.
}
%------------------------------------------------------------------------------------------------


%-------------------------------------------------------------------------------------------------------------------------------------
\subsubsection{Basics of Knot theory}
%-------------------------------------------------------------------------------------------------------------------------------------
  \begin{frame}[fragile]{Basics of \emph{knots theory} \qquad\hyperlink{frame:hydro3}{\beamerreturnbutton{}}}
		\begin{columns}
			\begin{column}[T]{.4\linewidth}
				\begin{figure}
					\caption{Reidmeister's move}
				\ifHandout
				%	\includestandalonewithpath[width=\textwidth,keepaspectratio]{../MS-Knots/Pictures}{Figure_trefoil}
				\else
				%	\includestandalonewithpath[width=\textwidth,keepaspectratio]{../MS-Knots/Pictures}{Animation_trefoil}
				\fi
				\end{figure}
			\end{column}
				\hfill
			\begin{column}[T]{.6\linewidth}
				\begin{enumerate}
					\item Knots = compact submanifolds of codimension 2 embedded in $\mathbb{R}^3$
						\begin{defblock}[n-link]
							%Disjoint union is written using \coprod, since it is in fact the coproduct in the category of sets.(\amalg)
							$\gamma: {\displaystyle\coprod_1^n S^1 } \to \mathbb{R}^3 $ embedding.				
						\end{defblock}
					\item studied \emph{modulo "ambient isotopies"}
						\begin{defblock}[Ambient isotopies]
							%Disjoint union is written using \coprod, since it is in fact the coproduct in the category of sets.(\amalg)
							$h : {\displaystyle\coprod_1^n S^1 }\times [0,1] \to \mathbb{R}^3 $ smooth homotopy
							s.t. $\hat{h}(t) : {\displaystyle\coprod_1^n S^1 } \to \mathbb{R}^3$ is an embedding $\forall t$	.				
						\end{defblock}	
					\item Holy grail of knot theory: classify all non equivalent (by ambient isotopies) n-links.
					\item \alert{These classes are invariant w.r.t. volume preserving diffeos.}
				\end{enumerate}
				%			
  	  \end{column}
    \end{columns}
	%
 \end{frame}
\note{
	\begin{itemize}
		\item It is possible to build a bridge between knot theory and multisymplectic geometry exploiting the tight connection between both of them with hydrodynamics.\\
		The cornerstone of this relationship is the ubiquitous role of the group of volume preserving diffeomorphisms.
		\item Knot  theory is not simply an instance of studying the topology of loops but the ambient space takes a significant role"
		\item Studying smooth knots is equivalent to study \href{http://mathworld.wolfram.com/PolygonalKnot.html}{tamed polygonal knots}.
		\item choosing a smooth parametrization is tantamount to fix an orientation of the knot.
	\end{itemize}
}
%-------------------------------------------------------------------------------------------------------------------------------------

%-------------------------------------------------------------------------------------------------------------------------------------
\subsubsection{Intermezzo: Poincare duals}
%------------------------------------------------------------------------------------------------

%------------------------------------------------------------------------------------------------
\begin{frame}{Recall: De Rham currents \qquad\hyperlink{frame:hydro3}{\beamerreturnbutton{}}}
	On smooth manifolds is possible to mimick the basic definition of distribution using differential forms:
		\begin{defblock}[De Rham k-Currents]
			\begin{displaymath}
				\mathcal{D}_k(M)
				:= 
			\biggr\{\eta: \Omega^k_c(M)\rightarrow \mathbb{R} \; \left\vert \; \text{\stackanchor{(seq.) continuous}{linear functionals}} \right\} 
				\cong
				\left(\Omega^k_c (M) \right)^\ast
			\end{displaymath}
		\end{defblock}
		%
		\vspace{-2.5ex}
		%
  	\onslide<2->{
  	\begin{columns}
		\begin{column}[t]{.5\linewidth}	
			\begin{defblock}[Annihilation set of $T \in \mathcal{D}$]
				Open subset $A \subset M$ s.t. \\ 
				$\langle T, \phi \rangle = 0 \quad \forall \phi 	
				\text{ s.t. } \text{supp}(\phi) \subset A$
			\end{defblock}
		\end{column}
		\begin{column}[t]{.5\linewidth}			
			\begin{defblock}[Support of $T \in \mathcal{D}$]
				$\text{supp}(T)$ = complement of the union of all open annihilation sets of $\eta$
			\end{defblock}
		\end{column}
  	\end{columns}
  	}		
%
  	\onslide<3->{
	  	We have the analogue of \emph{regular distributions}
			\begin{alignat*}{2}
			  D:\Omega^k(M) & \longrightarrow & \mathcal{D}^{n-k}(M) & \\
			  \eta & \longmapsto & D_\eta & \quad:\quad 
			  \langle D_\eta, \text{\textvisiblespace} \rangle = 
			  \int_M \blank \wedge \eta \\
			  \text{d}\sigma& \longmapsto & D_{\text{d}\sigma} & \quad:\quad 
			  \langle D_{\text{d}\sigma}, \blank \rangle = 
			  (-)^{k} \int_M   \text{d}\blank \wedge\sigma
			\end{alignat*}	  	
  	}				
		%
		\vspace{-2.5ex}
		%
  	\onslide<4->{
		\begin{defblock}[De Rham boundary operator]
			\begin{displaymath}
				\partial : \mathcal{D}_k(M) \rightarrow \mathcal{D}_{k-1}(M) \qquad \text{s.t.} \quad
				\langle \partial T, \blank \rangle = (-)^{k} \langle T, \text{d}\blank \rangle
			\end{displaymath}
		\end{defblock}
  	}							
\end{frame}
\note[itemize]{
	\item this is the analogue of the theory of distribution on $\mathbb{R}^n$
	\item Continuity in the sense of distributions means \emph{sequentially continuous} i.e
		If a sequence $\omega_{k}$ of smooth forms, all supported in the same compact set, is such that all derivatives of all their coefficients tend uniformly to 0 when 
		$k$ tends to infinity, then $T(\omega_{k})$ tends to 0.
	\item $\Omega^k_c(M)$ stands for compact supported k-forms.
	\item Recall the definition of \emph{support} of a differential form
		$$
			\text{supp}(\omega) := \overline{\lbrace p \in M \; \vert \: \omega_p = 0 \rbrace}
		$$
	\item from the definition of regular distribution we obtain the definition of boundary operator.	
	\item De Rham distributions build up a chain complex which is dual (modulo a sign) to the de Rham co-chain complex.
}
%------------------------------------------------------------------------------------------------



%------------------------------------------------------------------------------------------------
\begin{frame}{Recall:Poincaré duals \qquad\hyperlink{frame:hydro3}{\beamerreturnbutton{}}}
	\label{frame:poinduals}
	Given a compact, oriented embedded k-dim submanifold $\Sigma$ of $M$ (n-dim)
	\begin{displaymath}
		\left( i : 	\Sigma \hookrightarrow M \right) \in \text{Emb}_c(k)
	\end{displaymath}		
	you can associate a compactly supported \emph{DeRham current} $D_\Sigma$ defined as
	\begin{displaymath}
		\langle D_\Sigma, \omega\rangle = \int_\Sigma i^\ast (\omega) \qquad 
		\forall \omega \in \Omega^k
	\end{displaymath}
		%
		\vspace{-2.5ex}
		%
	\onslide<2->{
  	\begin{columns}
		\begin{column}[c]{.5\linewidth}	
			\begin{claimblock}
				$\partial D_\Sigma = (-)^{k-1} D_{\partial \Sigma}$ 
			\end{claimblock}
		\end{column}
		\begin{column}[c]{.5\linewidth}			
			\begin{claimblock}
				$ D_{\Sigma_1} \wedge D_{\Sigma_2} = D_{\Sigma_1 \cap \Sigma_2}$
			\end{claimblock}
		\end{column}
  	\end{columns}
  }		
	%
	\onslide<3->{
		\begin{table}[]
		\begin{tabular}{lll}
			Analogue of the Dirac delta function localized on $\Sigma$ & $\Rightarrow$ & 
			\alert{\faWarning \quad not regular \quad \faWarning}
		\end{tabular}
		\end{table}

		We're interested in a \emph{regular approximation} (regularization)
		\begin{defblock}[a (smooth) Poincaré dual of $\Sigma$]
		 $\eta_\Sigma \in \Omega^k$ supported on a tubular neighbourhood $T$ of $\Sigma$ s.t.
		 \begin{displaymath}
				\langle D_{\eta_\Sigma},\omega\rangle \equiv \int_M \omega \wedge \eta_\Sigma =
				\int_T i^\ast \omega \sim 
				\langle D_{\Sigma},\omega\rangle 
		 \end{displaymath}
		\end{defblock}		
		  	\centering \alert{\faWarning  \; ( Not unique! ) \; \faWarning }

  }		
\end{frame}
\note[itemize]{
	\footnotesize
	\item $\forall$ compact submanifold (dim=k) of $M$ (dim= n) one can associates a unique
	 de Rham current. 
	They can be understood as \emph{generalized} differential $(n-k)$-forms concentrated on $\Sigma$.
	This is the analogue of the Dirac delta function localized on $\Sigma$.
	\item Usual definition of Poincar\'e dual (in algebraic topology) is different:
	\\
		%\begin{defblock}[Poincaré dual of $\Sigma$]
			Unique $[\eta_\Sigma ] \in H^{n-k}_c(M)$ s.t.
			\begin{displaymath}
				\int_\Sigma i^\ast \omega = \int_M \omega \wedge \eta_\Sigma \qquad \forall [\omega] \in H^k(M)
			\end{displaymath}
		%\end{defblock}
		More conceptually, Poincaré duals can be seen as \emph{Thom Classes}.
	
	\item Proof of claim 1 is direct:
		\begin{displaymath}
			(-)^{k-1} \langle \partial D_\Sigma , \omega \rangle = 
			\langle D_\Sigma, d\omega\rangle =
			\int_\Sigma i^\ast d \omega = 
			 \int_\Sigma d i^\ast \omega = 
			 \int_{\partial \Sigma} i^\ast \omega =
			\langle D_{\partial \Sigma}, \omega \rangle
		\end{displaymath}
	\item Claim 2 is better understood with smooth Poincarè duals:
		\begin{displaymath}
			\text{supp}(\eta_1 \wedge \eta_2) \subset
			\text{supp}(\eta_1) \cap \text{supp}(\eta_2) \subset
			T_{\Sigma_1 \cap \Sigma_2} \quad\Rightarrow\quad
			\eta_1 \wedge \eta_2 = \eta_{\Sigma_1 \cap \Sigma_2}	
		\end{displaymath}				
		An example: 
		Take as $\Sigma_1$ the $z$-line in $\mathbb{R}^3$, $\eta_1 = \delta_{\{x=y=0\}}	dx \wedge dy$, and as $\Sigma_2$ the $xy$-plane, $\eta_2= \delta_{\{z=0\}} dz$.
		You get $\eta_1 \wedge \eta_2 = \delta_{\{x=y=z=0\}} dx \wedge dy \wedge dz = \eta_{\Sigma_1 \cap \Sigma_2}$
}
%------------------------------------------------------------------------------------------------



%------------------------------------------------------------------------------------------------
\begin{frame}{Relation with Gauss linking number \qquad\hyperlink{frame:hydro3}{\beamerreturnbutton{}}}
		%
		\vspace{-2.5ex}
		%
  	\begin{columns}
		\begin{column}[t]{.5\linewidth}	
			\begin{defblock}[Chern-Simons 3-form]
				$$
					CS({L}) :=  v_{L} \wedge  \omega_{ L} 
				$$
			\end{defblock}
		\end{column}
		\begin{column}[t]{.5\linewidth}	
			\begin{defblock}[Helicity]
				$$
 					{\mathcal H}(L) = \int_{\mathbb{R}^3} CS({L})
				$$
			\end{defblock}						
		\end{column}
  	\end{columns}
	\pause
	\begin{propblock}[
		Choosing a parametrization $\mathbf{r}_i$ (in standard coordinates) for each $L_i$ 
		\begin{displaymath}
			{\mathcal H}(L)  = 
			\sum_{i,j=1}^n
			\,\frac{1}{4\pi}
			\oint_{\gamma_i}\oint_{\gamma_j}
			\frac{\mathbf{r}_i - \mathbf{r}_j}{|\mathbf{r}_i - \mathbf{r}_j|^3}
			\cdot (d\mathbf{r}_i \times d\mathbf{r}_j) =
			\sum_{i,j=1}^n \ell(i,j)
		\end{displaymath}
			$\bullet$ $\ell(i,j) = \ell(j,i)$ : Gauss linking number of components $L_i$ and $L_j$ if $i\neq j$\\
			$\bullet$ $\ell(j,j)$ : {\it framing} of $L_j$\\ 
			\phantom{-------}\footnotesize{ (i.e. $\ell(L_j, L_j^{\prime})$ with $L_j^{\prime}$ being a section of the normal bundle of $L_j$.)}\normalsize

		]
	\pause
  	\begin{columns}
		\begin{column}[c]{.7\linewidth}	
		\underline{Sketch}: Cosider a Hopf Link
			\begin{displaymath}
				L(C,C') = \eta_C \wedge \eta_{\Sigma'} + \eta_{C'} \wedge \eta_\Sigma =
				\eta_{P'} + \eta_{P}
			\end{displaymath}
		\end{column}
		\begin{column}[c]{.25\linewidth}	
			\centering{
			\includegraphics[width=0.75\linewidth]{../MS-Knots/Pictures/GaussLink}
			}
		\end{column}
  	\end{columns}
			Therefore $\int L(C,C') = \ell(C,C')$ is counting the times that a knot cross another Seifert surface with sign given by the orientation.
		\end{propblock}
		
\end{frame}
\note[itemize]{
	\item Our previous construction is heavily dependent on a lot of choices 
	but we end up with a quantity that only depends on the starting n-link.
	\item ${\mathcal H}(L)$ is invariant under ambient isotopies.\\
	However, non ambient isotopic links do not necessarily yield different linking numbers
	(it is not an universal invariant!).
  	\begin{columns}
		\begin{column}[c]{.5\linewidth}	
			\centering{
			%\includegraphics[width=0.5\linewidth]{../Ms-Knots/Pictures/UnknotsGauss}
			}
		\end{column}
		\begin{column}[c]{.5\linewidth}	
			\centering{
			\includegraphics[width=0.35\linewidth]{../MS-Knots/Pictures/WhiteheadGauss}
			}
		\end{column}
  	\end{columns}	
	\item
		\begin{displaymath}
	\begin{split}
		\text{link}(\gamma_1,\gamma_2) &=\,\frac{1}{4\pi}
		\oint_{\gamma_1}\oint_{\gamma_2}
		\frac{\mathbf{r}_1 - \mathbf{r}_2}{|\mathbf{r}_1 - \mathbf{r}_2|^3}
		\cdot (d\mathbf{r}_1 \times d\mathbf{r}_2)\\[4pt]
		 &= \frac{1}{4\pi}\int_{S^1 \times S^1} \frac{\text{det}(\dot{\gamma_1}(s),
	 \dot{\gamma_2}(t),\gamma_1(s)-\gamma_2(t))}{|\gamma_1(s)-\gamma_2(t)|^3}\, ds \, dt
	\end{split}
	\end{displaymath}			
}
%------------------------------------------------------------------------------------------------

%------------------------------------------------------------------------------------------------
\begin{frame}{Cohomological interpretation of the linking number \qquad\hyperlink{frame:hydro3}{\beamerreturnbutton{}}}\label{frame:highorderlinking}
		\vspace{2ex}
  	\begin{columns}
		\begin{column}[c]{.75\linewidth}	
			$\bullet$ Choose a pair of linked knots (part of a more complex link).
			Define
			$$ \Xi_{1 2} := - v_{L_1} \wedge v_{L_2} \in \Omega^2(\mathbb{R}^3)$$
			you get:
			\begin{displaymath}
				\begin{split}
				d \Xi_{1 2} =& -\omega_1 \wedge v_2 + v_1 \wedge \omega_2 = CS({L_1\cup L_2}) - 
				\text{"\stackanchor{self}{linking}"}
				\\
				&\Rightarrow \int d \Xi_{1 2} = \ell(1,2)
				\end{split}
			\end{displaymath}		
		\end{column}
		\begin{column}[c]{.20\linewidth}	
			\centering{
			\includegraphics[width=\linewidth]{../MS-Knots/Pictures/WhiteheadGauss}
			}
		\end{column}
  	\end{columns}	
	\pause	
	\vspace{1ex}
	$\bullet$ While $\Xi_{1 2}$ is not uniquely defined,it determines an unique class in the \emph{cohomology of the link}
	(independent of the choices)
	\begin{displaymath}
		\langle L_1, L_2 	\rangle := 
		\left[\Xi_{1 2}\big\vert_{S^3\setminus L} \right] \in H^2(S^3\setminus L)
	\end{displaymath}
	\pause
	\vspace{-2ex}
  	\begin{columns}
		\begin{column}[c]{.7\linewidth}	
			\begin{propblock}[$\ell(1,2)= 0 \Leftrightarrow \langle L_1, L_2 \rangle = 0$]
				\vspace{-4ex}
				\begin{displaymath}
				\begin{split}
					l(1,2)=0 ~&\Leftrightarrow~ d \Xi_{1 2} = 0 \in \Omega^2(\mathbb{R}^3)\\
					\text{"Poincar\'e lemma"}	
					~&\Rightarrow~ \exists v_{1 2} \;:\; d v_{1 2} = \Xi_{ 1 2}	\\
					[ \Xi_{1 2} ] = 0 \in H^2(\mathbb{R}^3) ~&\Rightarrow~
					\left[\Xi_{1 2}\big\rvert_{S^3\setminus L} \right] = 0 \in H^2(S^3\setminus L)
				\end{split}
				\end{displaymath}		
			\end{propblock}
		\end{column}
		\begin{column}[c]{.4\linewidth}	
	  	\begin{asideblock}[Cohomology groups of a n-Link]%Shades of...
  			\begin{table}[] % http://tablesgenerator.com/
					\begin{tabular}{l}
						$H^0 (S^3 \setminus L) \cong {\mathbb R}$ \\
						$H^1 (S^3 \setminus L) \cong {\mathbb R}^{n}$ \\
						$H^2 (S^3 \setminus L) \cong {\mathbb R}^{n-1}$ \\
						$H^3 (S^3 \setminus L) \cong 0$
					\end{tabular}
				\end{table}
	  	\end{asideblock}
		\end{column}
  	\end{columns}			



	

\end{frame}
\note[itemize]{
	\item The cohomology of a n-link is the de Rham cohomology of $S^3 \setminus L$, 
	where $S^3$ has to be understood as the compactification of the Euclidean space
	\item Being $\Xi_{1 2}$ closed follows from the fact that 
	$\text{supp}(d \Xi_{1 2}) \subseteq L$ \\
	Similarly, you get that all the velocity 1-forms $v_i$ are closed.
	The cohomology classes of these forms, one for each component of the link, 
	are precisely the generators of $H^1 (S^3 \setminus L)$.
	\item Why such a class is called a number? 
	The name follows from the simplest case of $n=2$.
	\item \underline{Upshot:} we can associate to any pair of knots in a link a  cohomology class.
}
%------------------------------------------------------------------------------------------------

%------------------------------------------------------------------------------------------------
\begin{frame}[shrink]{Higher order linking numbers \qquad\hyperlink{frame:hydro3}{\beamerreturnbutton{}}}
	\begin{itemize}
		\item[•] Take a link with 3 or more components
		\item[•] 	assume all ordinary mutual linking numbers vanish: $\langle L_i, L_j \rangle =0$
		\item[•] Out of the primitives obtained in the previous proposition you can manufacture another closed 2-form
	\end{itemize}
		\vspace{-2ex}
  	\begin{columns}
		\begin{column}[c]{.5\linewidth}	
			\begin{defblock}[Massey product]
				$\Xi_{1 2 3} = - v_1 \wedge v_{2 3} - v_{1 2} \wedge v_3 
				\in \Omega^2(\mathbb{R}^3)$
			\end{defblock}
		\end{column}
		\begin{column}[c]{.5\linewidth}	
			\begin{defblock}[Third order linking number (class)]
				$\langle L_1, L_2, L_3 \rangle := 
				\left[\Xi_{1 2 3}\big\rvert_{S^3\setminus L} \right]
				\in H^2(S^3\setminus L)$
			\end{defblock}
		\end{column}
  	\end{columns}			
  	\pause
	The procedure can be iterated (obstructed by the non vanishing of a higher linking number) yielding a hierarchy of pairs
	\begin{displaymath}
		\Xi_I \in \Omega^2 \qquad v_I \in \Omega^1
	\end{displaymath}
	\center
	\footnotesize{(I = multi index constructed out of the set $\{1,\ldots,n\}$ of the $n$-link components)}\normalsize 

	\footnotesize{
	\pause 	\center 	\textbf{-- Applications --}
  	\begin{columns}
		\begin{column}[c]{.45\linewidth}
			Distinguish different links with same (vanishing) Gauss linking number	
			\center
			\begin{tikzpicture}
			\node[inner sep=0pt] (A) at (0,0)
			    {\includegraphics[width=.35\textwidth]{../MS-Knots/Pictures/UnknotsGauss}};
			\node[inner sep=0pt] (B) at (3,0)
			    {\includegraphics[width=.35\textwidth]{../MS-Knots/Pictures/WhiteheadGauss}};
			\draw[draw=none] (A) -- (B)
			    node[midway,fill=white] {$ \not\sim $};
			\end{tikzpicture}
		\end{column}
		\hspace{2ex}
		\begin{column}[c]{.45\linewidth}	
			Ascertain \emph{Brunnian character}
			\centering{
			\includegraphics[width=0.4\linewidth]{../MS-Knots/Pictures/BorromeanLink}
			}
		\end{column}
  	\end{columns}			
	}


\end{frame}
\note[itemize]{
	\item $\Xi_I$ and $v_I$ can be again interpreted via Poincar\'e duality as the smooth Poincar\'e dual of an auxiliary trivial knot and of a corresponding Seifert surface respectively.
	\item Below, two possible application of this machinery are mentioned .
	\item This procedure can be used as a tool to distinguish knots with vanishing linking number. (Figure: two unknots versus a Whithead Link. The latter involve forth order linking numbers computing admitting indices repetition).
	\item Ascertain the Brunnian character of a $n$-link. A link is \emph{Brunnian} 
		or \emph{almost trivial} when it becomes trivial upon removing any component. 
		(Figure: Borromean link).
}
%------------------------------------------------------------------------------------------------

%------------------------------------------------------------------------------------------------
\begin{frame}{Massey products as Conserved quantities \qquad\hyperlink{frame:hydro3}{\beamerreturnbutton{}}}
	%
	\begin{propblock}[
		\quad(1) $\Xi_I$ exact $\Rightarrow$ $v_I$ are hamiltonian (w.r.t volume form) \\
		\phantom{-------------}(2) Massey 2-forms $\Xi_I$ are globally conserved
		]
		\vspace{-4ex}
		\begin{displaymath}
		\begin{split}
			(1)&\quad d v_{I} = \Xi_{I} = \iota_{\xi_{I}} \text{Vol}_{\mathbb{R}^3} \qquad 
			\text{defining}\quad \xi_{I} = \alpha^{-1}(\Xi_{I})
			\qquad \Rightarrow \text{$v_{123}$ Hamiltonian}
			\\
			(2)&\quad \Xi_I \text{ closed} \quad \Rightarrow \quad \mathcal{L}_\xi \Xi_I = d \iota_xi \Xi_I \in B^2
		\end{split}
		\end{displaymath}

	\end{propblock}	
	By construction: the momenta associated to the divergence-free field $\xi_{I}$
	 correspond to $v_I$
		\begin{displaymath}
			v_I = f_1(\xi_I)
		\end{displaymath}
	\pause \vfill
	\begin{propblock}[
		The 1-forms $v_I$ are {\rm first integrals in involution} with respect to the flow generated by the 
		Hamiltonian vector field $\xi_{ L}$, i.e.
		\begin{itemize}
			\item ${\mathcal L}_{\xi_{ L}} v_I = 0$ ($v_I$'s are  {\rm strictly conserved})
			\item $\{v_I, v_J \} = 0$ (for multiindices $I$ and $J$)
		\end{itemize}
	]


	See Thm 6.2 \href{https://arxiv.org/abs/1805.01696}{arXiv: 1805.01696}
	\end{propblock}	
	
	

\end{frame}
\note[itemize]{
	\item $\xi_{123} = \alpha^{-1}(\Xi_{123})$, constructed resorting on the Hodge machinery of $\mathbb{R}^3$ can be regarded as the vorticity field concentrated on a auxiliary knot.
	\item {\bf Proof. Thm 6.2} Using Cartan's formula, we get
		$$
			{\mathcal L}_{\xi_{ L}} v_I = d \iota_{\xi_{ L}} v_I + \iota_{\xi_{ L}} d v_I = 
			d \iota_{\xi_{ L}} v_I  -  \iota_{\xi_{ L}} \iota_{\xi_{ I}} \nu,
		$$
		but the second summand vanishes in view of the general expression
		$$
			\{v_\xi, v_\eta \}(\cdot) = \nu (\xi, \eta, \cdot)
		$$
		and of the peculiar structure of the vector fields involved (they either partially coincide or have disjoint supports). 
		By the same argument, one gets $\iota_{\xi_{ L}} v_I = 0$. From that the the {\it strict} conservation of the $v_I$'s is immediate.\par
	\item  Poincar\'e dual interpretation $\Rightarrow$ $\iota_{\xi_{ L}} v_L = 0$
		$\Rightarrow$ ${\mathcal L}_{\xi_{ L}} v_L = 0 $
		\\
		(this is {\it not} to be expected a priori in multisymplectic geometry)


}
%------------------------------------------------------------------------------------------------










%------------------------------------------------------------------------------------------------
\end{document}
