%+------------------------------------------------------------------------+
%| Frame: Lie infinity algebra Definition
%| Author: Antonio miti
%+------------------------------------------------------------------------+

\documentclass{standalone}
\usepackage{tikz}
\usepackage{verbatim}
\usetikzlibrary{arrows,shapes}
\usepackage{amsfonts}


\begin{document}
	\tikzstyle{every picture}+=[remember picture]
	\everymath{\displaystyle}
	\tikzstyle{na} = [baseline=-.5ex]
	\begin{minipage}[c]{\linewidth}
		\begin{minipage}[t]{0.3\linewidth}
			\vspace{1em}
			\begin{displaymath}
								 (
								 \tikz[baseline]{
								            \node[fill=blue!20,anchor=base] (t1)
								            {$ L$};
								        } 
									,
									 \tikz[baseline]{
								            \node[fill=blue!20,anchor=base] (t2)
								            {$ \lbrace \mu_i \rbrace_{i \in \mathbb{N}} $};
									}
								)
				\end{displaymath}
				%	
		\tikz[na] \node[scale=0.5,coordinate,fill=blue!20,draw,circle] (n1) {};		    
				 Graded vector space $$L = \bigoplus_{i\in\mathbb{Z}} L_i$$		
		\end{minipage}
		%
		\begin{minipage}[t]{0.7\linewidth}
			\tikz[na]\node [scale=0.5,coordinate,fill=blue!20,draw,circle] (n2) {};	    
				  	Family of multilinear maps (\emph{multi-brackets}) $\mu_k : L^{\otimes k} \rightarrow L$
				\vspace{-0.5em}
				\begin{itemize}
				    	\item[-] graded skew-symmetric
				    	\item[-] $\textrm{deg}(\mu_k)= k - 2$
				    		%\footnote{i.e.$\textrm{deg}\left(l_k(x_1,\ldots,x_k)\right)= \sum_i \textrm{deg}(x_i) +2 -k $}
							\item[-] 	satisfying \emph{"Higher Jacobi"} relations  
							$$\displaystyle J_m=\sum_{i+j=m+1}	\mu_i \triangleleft \mu_j = 0
							\quad 							(\forall m\geq 1)$$		
				\end{itemize}		
		\end{minipage}

		\footnotetext{Notation: $\qquad\mu_i\triangleleft \mu_j := \frac{(-)^{i(j+1)}}{j!(i-1)!}\mu_i\circ(\mu_j\otimes id_{i-1}) \circ \mathcal{A}~, \quad \mathcal{A} =$ (graded) total skew-symmetrizator.}
	\end{minipage}




% Now it's time to draw some edges between the global nodes. Note that we
% have to apply the 'overlay' style.
\begin{tikzpicture}[overlay]
        \path[->] (n1) edge [bend left] (t1);
        \path[->] (n2) edge [bend right] (t2);
       % \path[->] (n3) edge [out=0, in=-90] (t3);
\end{tikzpicture}

\end{document}
