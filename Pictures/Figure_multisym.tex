%+------------------------------------------------------------------------+
%| Frame: Multisymplectic Manifold Definition
%| Author: Antonio miti
%+------------------------------------------------------------------------+

\documentclass[border=3pt]{standalone}
\usepackage{tikz}
\usepackage{verbatim}
\usetikzlibrary{arrows,shapes}


\begin{document}
% For every picture that defines or uses external nodes, you'll have to
% apply the 'remember picture' style. To avoid some typing, we'll apply
% the style to all pictures.
\tikzstyle{every picture}+=[remember picture]

% By default all math in TikZ nodes are set in inline mode. Change this to
% displaystyle so that we don't get small fractions.
\everymath{\displaystyle}

	\begin{minipage}[c]{0.25\textwidth}
			\begin{displaymath}
				 \big(
				 \tikz[baseline]{
				            \node[fill=blue!20,anchor=base] (t1)
				            {$ M$};
				        } 
					,
					 \tikz[baseline]{
				            \node[fill=blue!20,anchor=base] (t2)
				            {$ \omega$};
					}
				\big)
			\end{displaymath}	
	\end{minipage}
	\begin{minipage}[t]{0.75\textwidth}
		\tikzstyle{na} = [baseline=-.5ex]
		%
		\tikz[na] \node[scale=0.5,coordinate,fill=blue!20,draw,circle] (n1) {};		    
		   		 Smooth Mfd.
		 \\
		 \tikz[na]\node [scale=0.5,coordinate,fill=blue!20,draw,circle] (n2) {};	    
		    		non-degenerate, closed, $(n+1)$-form.
	\end{minipage}

% Now it's time to draw some edges between the global nodes. Note that we
% have to apply the 'overlay' style.
\begin{tikzpicture}[overlay]
        \path[->] (n1) edge [bend right] (t1);
        \path[->] (n2) edge [bend left] (t2);
       % \path[->] (n3) edge [out=0, in=-90] (t3);
\end{tikzpicture}

\end{document}